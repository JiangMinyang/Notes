\documentclass[12pt, a4paper]{article}
\usepackage[letterpaper, hmargin=0.75in, vmargin=0.75in]{geometry}
\usepackage{graphicx}
\usepackage[hyphens]{url}
\usepackage{hyperref}
\usepackage{listings}
\usepackage{pgf}
\usepackage{tikz}
\usepackage{courier}
\usepackage{amsfonts,amssymb,amsmath,amsthm,lastpage,fancyhdr,wrapfig,multirow}
\usepackage{palatino}
\usepackage{amsmath}
\usepackage[english]{babel}
\usepackage[utf8]{inputenc}
\usepackage{fancyhdr}
\textwidth 7.5in
\oddsidemargin -.5in
\topmargin -0.70in
\textheight 9.8in                      

\pagestyle{fancy}

%**************Fill in your ID and initials here*****************
\newcommand{\mc}[1]{\ensuremath{\mathcal{{#1}}}}
\newcommand{\mb}[1]{\ensuremath{\mathbb{{#1}}}}
\newcommand{\mf}[1]{\ensuremath{\mathfrak{{#1}}}}
\newcommand{\N}[1]{\ensuremath{\{1,\ldots,{#1}\}}}

\newcommand{\Worth}[1]{\{{#1} marks\}}
\newcommand{\Sln}{\smallskip \textbf{Solution.} }
\newcommand{\Extra}[1]{\{Extra credit: {#1} marks\}}


\setlength{\parskip}{0.15in}
\setlength{\parindent}{0in}


\newcommand{\NP}{\newpage \vspace*{-0.4in}}
\newcommand{\FP}{\vspace*{-0.6in}}
\newcommand{\tab}[1][1cm]{\hspace*{#1}}
\newcommand{\ES}{Erwin Schr\"odinger}

\lstset{ %
basicstyle=\ttfamily\scriptsize,commentstyle=\scriptsize\itshape,showstringspaces=false,breaklines=true}

\tikzstyle{input} = [coordinate]
\tikzstyle{output} = [coordinate]
\tikzstyle{joint} = [draw, circle, minimum size=1em]
\tikzstyle{block} = [draw, rectangle, minimum height=3em, minimum width=6em]


\title{\huge SE380 Notes}
\author{Minyang Jiang}
\date{\today}

\begin{document}

\maketitle

\NP
\section*{1.2 Introduction}
\subsection*{What is control engineering?}
Example (automated highway)
% \begin{center}
%   \begin{tikzpicture}[auto, node distance=2cm, >=stealth]
%     \node [input, name=input] (input) {};
%     \node [joint, right of=input] (joint1) {};
%     \node [block, right of=joint1] (controller) {Controller};
%     \node [block, right of=controller, node distance=4cm] (algorithm) {System};
%     \draw [->] (controller) -- node[name=u] {$u$} (algorithm);
%     \node [block, below of=u] (measurement) {Measurement};
%     \node [output, right of=algorithm] (output) {};

%     \draw [->] (input) -- node{$R(s)$} (joint1);
%     \draw [->] (joint1) -- node{} (controller);
%     \draw [->] (algorithm) -- node[name=y]{$Y(s)$} (output);
%     \draw [->] (y) |- node{} (measurement);
%     \draw [->] (measurement) -| node[pos=0.99]{$-$} node [near end]{$y_m$} (joint1);

%   \end{tikzpicture}
% \end{center}


Ex. web server sec1.4.1
Control Objectives
\begin{enumerate}
  \item Don't let queue length set too long
  \item Don't let queue length get to zero
\end{enumerate}
We need to keep the queue length some known value r(t)\\
the difficulty is that the \underline{service rate} is not known - depends on many things.\\
e.g. number of clients - model as a disturbances d(t)\\
We must decide on the \underline{request rate} $u(t)$ based on $r(t)$ and $y(t)$


\begin{center}
  \begin{tikzpicture}[auto, node distance=2cm, >=stealth]
    \node [input, name=input] (input) {r};
    \node [joint, right of=input] (joint1) {};
    \node [block, right of=joint1] (controller) {Controller};
    \node [joint, right of=controller] (joint2) {};
    \node [input, above of=joint2] (disturbance) {d};
    \node [block, right of=joint2, ] (server) {Server};
    \node [input, below of=joint2] (measurement) {};
    \node [output, right of=server, node distance=3cm] (output) {};

    \draw [->] (input) -- node{$R(s)$} (joint1);
    \draw [->] (joint1) -- node{} (controller);
    \draw [->] (controller) -- node{} (joint2);
    \draw [->] (joint2) -- node{} (server);
    \draw [->] (disturbance) -- node [pos=0.01]{$d$} (joint2);
    \draw [->] (server) -- node[name=y]{$Y(s)$} (output);
    \draw [-] (y) |- node{} (measurement);
    \draw [->] (measurement) -| node [near end]{$y_m$} (joint1);
  \end{tikzpicture}
\end{center}

Control engineering attempts to change the behavior of a system (plant) in a useful way dispits the present of external influences (disturbances) and despite model uncertainty\\
We change the behavior of the plant by connecting it to another system (controller) Feedback is the most powerful inter-connector strategy\\
Control system
\begin{enumerate}
  \item sense the operation of system
  \item compose against a desired behavior
  \item compute a corrective action inforced by a model of the system's response to the external stimul;
  \item Actuate the system to effect demand change
\end{enumerate}
\section*{1.3 Control Engineering Design}
\begin{enumerate}
  \item Study system to be controlled; decide on sensors and actuators
  \item model resulting system
  \begin{enumerate}
    \item mathematical model
    \item often one or more differential equations\\
    e.g. follower $\frac{dx}{dt}=u$
  \end{enumerate}
  \item Simplify model if necessary
  Have a transfer function of the plant (input-output model)
e.g. follower. $$L\left \{\frac{dx_f}{dt} \right \}=L\left \{u \right\}\Rightarrow sX_f(s)=U(s)\Rightarrow\frac{X_f(s)}{U(s)}=\frac{1}{s}$$
a system has a TF if, and only if it is linear and time invariant
  \item Analyse resulting system
  \item Determine specifications stability, good steady behaviour robustness, good transrent performance
  \item Decide on type of controller
  \item Design controller
  \begin{enumerate}
    \item in this course, the controller itself is a TF
    \item this TF corresponds to a diff eq relating the inputs/outputs at the controller
  \end{enumerate}
  \item Simulate
  \item Return to step 1 if necessary
  \item Implement controller
\end{enumerate}
- the ODE from step 7 is approximated as discretized and difference equation and implemented in software\\
e.g. follower. $$u(t)=-K_p(r(kT)-y(kT)), kT\leq t\le (k+1)T$$
T: sampling period

\end{document}