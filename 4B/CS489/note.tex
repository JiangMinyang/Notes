\documentclass[letterpaper]{article}
\usepackage[letterpaper, hmargin=0.75in, vmargin=0.75in]{geometry}
\usepackage{graphicx}
\usepackage[hyphens]{url}
\usepackage{hyperref}
\usepackage{listings}
\usepackage{pgf}
\usepackage{tikz}
\usepackage{courier}
\usepackage{amsfonts,amssymb,amsmath,amsthm,lastpage,fancyhdr,wrapfig,multirow}
\usepackage{mathrsfs}
\usepackage{palatino}
\usepackage{amsmath}
\usepackage[english]{babel}
\usepackage[utf8]{inputenc}
\usepackage{fancyhdr}
\textwidth 7.5in
\oddsidemargin -.5in
\topmargin -0.70in
\textheight 9.8in                      

\pagestyle{fancy}

%**************Fill in your ID and initials here*****************
\newcommand{\mc}[1]{\ensuremath{\mathcal{{#1}}}}
\newcommand{\mb}[1]{\ensuremath{\mathbb{{#1}}}}
\newcommand{\mf}[1]{\ensuremath{\mathfrak{{#1}}}}
\newcommand{\N}[1]{\ensuremath{\{1,\ldots,{#1}\}}}

\newcommand{\Worth}[1]{\{{#1} marks\}}
\newcommand{\Sln}{\smallskip \textbf{Solution.} }
\newcommand{\Extra}[1]{\{Extra credit: {#1} marks\}}


\setlength{\parskip}{0.15in}
\setlength{\parindent}{0in}


\newcommand{\NP}{\newpage \vspace*{-0.4in}}
\newcommand{\FP}{\vspace*{-0.6in}}
\newcommand{\tab}[1][1cm]{\hspace*{#1}}
\newcommand{\ES}{Erwin Schr\"odinger}

\lstset{ %
basicstyle=\ttfamily\scriptsize,commentstyle=\scriptsize\itshape,showstringspaces=false,breaklines=true}

\tikzstyle{input} = [coordinate]
\tikzstyle{output} = [coordinate]
\tikzstyle{joint} = [draw, circle, minimum size=1em]
\tikzstyle{block} = [draw, rectangle, minimum height=6em, minimum width=6em]


\title{\huge CS489 (Neural Network) Notes}
\author{Minyang Jiang}
\date{\today}

\begin{document}

\maketitle

\section{Unit 1: Simulating Neurons}

Difference in charge across the membrane induces a voltage difference, and is called membrain potential.

\underline{Action Potential}: Neurons have a peculiar behaviour, they can produce a spike of electrical activity called an action potential. This electrical burst trvels along the neuron's axon to its synapses, it passes signals to other neurons.

Let $V$ be the membrane potential. A neuron usually keeps a membrane potential of around $-70mV$

The fraction of $K+$ channels that are open is $n^4$, where
\begin{align*}
	\frac{dn}{dt}=\frac{1}{t_n(v)}(n_{ob}(v)- n)\\
	t_n(v): \text{fine constant}\\
	n_ob: \text{equilibrium s}
\end{align*}















			
			
\end{document}