\documentclass{article}

\usepackage{amsfonts,amssymb,amsmath,amsthm,lastpage,fancyhdr,wrapfig,multirow}
\usepackage{graphicx}
\usepackage{palatino}

\textwidth 7.5in
\oddsidemargin -.5in
\topmargin -0.70in
\textheight 9.8in                      

\pagestyle{fancy}

%**************Fill in your ID and initials here*****************
\newcommand{\mc}[1]{\ensuremath{\mathcal{{#1}}}}
\newcommand{\mb}[1]{\ensuremath{\mathbb{{#1}}}}
\newcommand{\mf}[1]{\ensuremath{\mathfrak{{#1}}}}
\newcommand{\N}[1]{\ensuremath{\{1,\ldots,{#1}\}}}

\newcommand{\Worth}[1]{\{{#1} marks\}}
\newcommand{\Sln}{\smallskip \textbf{Solution.} }
\newcommand{\Extra}[1]{\{Extra credit: {#1} marks\}}


\setlength{\parskip}{0.15in}
\setlength{\parindent}{0in}


\newcommand{\NP}{\newpage \vspace*{-0.4in}}
\newcommand{\FP}{\vspace*{-0.6in}}

\begin{document}
COMPOSING POWER SERIES \\
consider the power series $A(x)=\sum_{n=0}^\infty nx^n$ and $B(x) = \sum_{n=0}^\infty (n^2+1)x^n$\\
What are the conefficients of $x^0$ and $x^1$ in $B(A(x))$ and $A(B(x))$\\
\begin{align*}
B(A(x))&=\sum_{n=0}^\infty \{(n^2+1)(A(x))^n\}\\
&=\sum_{n=0}^\infty\{(n^2+1)(\sum_{j=0}^\infty jx^j)^n\}\\
&=(1+0)x^0+(2*1)x^1+\hdots
\end{align*}
What is the constant term (cofficient of $x^1$) in each $(\sum_{j=0}^\infty jx^j)^n$?\\
$$\sum_{j=0}^\infty jx^j=0x^0+1x^1+2x^2+\hdots = x(1+2x+3x^2+\hdots)$$
$[x^0]\sum_{j=0}^\infty jx^j = 0,\ [x^0](\sum_{j=0}^\infty jx^j)^2=0$ for all $n\geq 1,\ [x^0](\sum_{j=0}^\infty jx^j)^n=0$\\
$$\sum_{n=0}^\infty \{(n^2+1)(\sum_{j=0}^\infty jx^j)^n\}=1*1+2*(\sum_{j=0}^\infty jx^j)^1 + 5*(\sum_{j=0}^\infty jx^j)^2+ 10*(\sum_{j=0}^\infty jx^j)^3+\hdots$$
coefficient of $x^1$:$=0+2*1_0+0+0+\hdots$\\

$$A(B(x))=\sum_{n=0}^\infty \{n(\sum_{j=0}^\infty(j^2+1)x^j)^n\}$$
$$\sum_{j=0}^\infty(j^2+1)x^j=1x^0+2*x^1+5*x^3+10x^3+\hdots$$
$$(\sum_{j=0}^\infty(j^2+1)x^j)^2=1*1x^0+4x^1+14x^2+\dots$$
$$[x^0](\sum_{j=0}^\infty(j^2+1)x^j)^n)=1$$
$$[x^1](\sum_{j=0}^\infty(j^2+1)x^j)^n)=2n$$
$$n=2: (1x^0+2*x^1+5*x^3+10x^3+\hdots)(1x^0+2*x^1+5*x^3+10x^3+\hdots)=1x^0+(1*2+2*1)x^1+\hdots$$
$$[x^0]A(B(x))=0*1+1*1+2+1+3*1+\hdots = \sum_{n_0}^\infty n*1$$

What is the cofficient of $x^2$ in $A(B(x))$ where $A(x)=\frac{1}{1-x}$ and $B(x)=x+2x^2$?\\
\begin{align*}
A(B(x))&=\frac{1}{1-(x+2x^2)}\\
&=\sum_{n=0}^\infty (x+2x^2)^2\\
&=\sum_{n=0}^\infty x^n(1+2x)^n
\end{align*}

\begin{align*}
[x^2]A(B(x))&=[x^2]\sum_{n=0}^2x^n(1+2x)^n\\
&=[x^2](1+x(1+2x)+x^2(1+2x)^2)\\
&=[x^2](1+x+2x^2+x^2+4x^3+4x^4)\\
&=3
\end{align*}

$A(x)=\frac{1}{1-3x+4x^3}$, Find a recurrence relation for the coefficients of $A(x)$\\
$a_n=a_{n-1}+a_{n-2}$

Let $A(x)=\sum_{n=0}^\infty a_nx^n$, Then $(1x^0-3x+4x^3)(\sum_{n=0}^\infty a_nx^n)=1x^0+0x+0x^2+\hdots$\\
$a_0x^0+(1a_1-3_0)x^1+(1a_2-3a_1)x^2+(1a_3-3a_2+4a_0)x^3+(1a_4-3a_3+4a_1)x^4+\hdots+(1a_n-3a_{n-1}+4a_{n-3})x^n+\hdots$



\section*{May 18 2016}

\section*{Product lamma}
Product lamma: Set $A$, $B$ with lengths $\alpha ,\beta$. Set $A\times B$ with weight $\omega(a,b)=\alpha(a)+\beta(b)$, Then $$\Phi_{A\times B}(x)=\Phi_A(x)\Phi_B(x)$$

Example: How many ways can a squence of k non-negative integers sum up to n?

[$k=4;\ n=10$    $(1,2,3,4)\ (0,0,5,5)$]\\
$N_0=\{0,1,2,3,4,\hdots\}$\\
Consider the set $N_0^k=\{(a_1,a_2,\hdots,a_k)|a_i \in N_0\}$\\
Define $\omega(a_1,a_2,\hdots,a_k)=a_1+a_2+\hdots+a_k$    Use $\alpha(a)=a$ for each $N_0$\\
Then product lemma applies: For $N_0$ with $\alpha$, $\Phi_{N_0}(x)=1+x+x^2+x^3+\hdots=\frac{1}{1-x}$\\
By product lemma, $\Phi_{N_0^k}(x)=(\Phi_{N_0}(x))^k=\frac{1}{(1-x)^k}$\\
Answer is: $$[x^n]\frac{1}{(1-x)^k}$$ This is $$\binom{n+k-1}{k-1}$$

Combinatorial proof of $\binom{n+k-1}{k-1}$\\
$k=4$, $n=10$: want $a_1+a_2+a_3+a_4=10$ \\
For each $k$ tuple $(a_1,a_2,\hdots ,a_k)$ is a non-neg soln to $a_1+a_2+\hdots +a_k=n$ \\
Consider binary strings with n $0$s and $k-1\ 1$s (dividers) \\
Form a bijection so that $(a_1,a_2,\hdots,a_k)$ is mapped to $0^{a_1}|0^{a_2}|\hdots |0^{a_k}$ where $0^{a_i}$ represents $a_i$ consecutive $0$s\\ This is reversible, So the \# of k-tuples is equal to \# of binary str with n $0$s and $k-1$ $1$s, which is $\binom{n+k-1}{k-1}$

\section*{Integer compositions}
Definition: A k-tuple $(a_1,\hdots,a_k)$ of positive integers is a composition of $n$ if $a_1+\hdots+a_k=n$. Such a composition has k parts\\
Example: composition of 5 includes $(1,3,1)$, $(5)$, $(2,3)$, $(3,2)$, $(1,1,1,1,1)$\\
Node: \par
1) parts are at least 1 \par
2) Order does matter \par
3) There is one composition of 0 which is $()$ \\
Example: How many compositions of $n$ have $k$ parts?\\
The set $N^k$ represents all compositions with $k$ parts \\
Let $\omega(a_1,\hdots,a_k)=a_1+\hdots +a_k$ \\
Use $\alpha(a)=a$ for each $N$\\
$$\Phi_N(x)=x+x^2+x^3+\hdots=\frac{x}{1-x}$$
So by product lemma:
$$\Phi_{N^k}(x)=(\Phi_N(x))^k=\left( \frac{x}{1-x} \right)^k=\frac{x^k}{(1-x)^k}$$

Our answer is:
\begin{align*}
&[x^n]\frac{x^k}{(1-x)^k}\\
=&[x^{n-k}]\frac{1}{(1-x)^k}\\
=&\binom{(n-k)+k-1}{k-1}\\
=&\binom{n-1}{k-1}
\end{align*}

\section*{May 25 2016}
1. Integer composition\\
2. Binary Strings\\
Gen ser for all compositions where every part is odd $$\frac{1-x^2}{1-x-x^2}$$
Let $a_n$ be the \# of comp of $n$ where every part is odd $$a_n=a_{n-1}+a_{n-2}\hspace{10pt} for\ n\geq 3$$
Combinatiorial proff of this recurrence:\\
Let $S_n$ be the set of all comp of $n$ where every part is odd\\
We need to prove that $|S_n|=|S_{n-1}|+|S_{n_2}|$ for $n\geq 3$ \\
We define a bijection $f: S_n \rightarrow S_{n-1} \cup S_{n-2}$ as follows:\\
$n=5$: $(5),(1,1,3), (1,3,1), (3,1,1), (1,1,1,1,1)$\\
$n=4$: $(1,3), (3,1), (1,1,1,1)$\\
$n=3$: $(3), (1,1,1)$\\
for $n-1$, add 1 to front\\
for $n-2$, add 2 to the first\\
For each $(a_1\hdots, a_k) \in S_n$ define
$$
f(a_1,\hdots,a_k)=
\begin{cases}
(a_2, \hdots, a_k) \hspace{10pt} (a_1=1) \rightarrow in S_{n-1}\\
(a_1-2,a_2,\dots, a_k) \hspace{10pt} (a_1 > 1) \rightarrow in S_{n-2}
\end{cases}
$$ 
Every part is odd after the mapping

The inverse is $f^{-1}:S_{n-1}\cup S_{n-2} \rightarrow S_n$ where for each $(b_1,\hdots,a_k) \in S_{n-1} \cup S_{n-2}$, define:
$$
f^{-1}(b_1,\hdots,b_k)=
\begin{cases}
(1,b_1, \hdots, b_k) \hspace{10pt} b_1+\hdots+b_k = n-1\\
(b_1+2,b_2,\dots, b_k) \hspace{10pt} b_1+\hdots+b_k = n-2
\end{cases}
$$ 
We can recursively build up $S_n$ based on $S_{n-1}$ and $S_{n-2}$ for each comp

\section*{Binary Strings}
Definitions: A binary string is a sequence of 0s and 1s.\\
The length of a string is the total \# of 0s and 1s\\
There is only one string of length 0, the emtpy string or null string, denoted $\epsilon$\\
The concatenation of two strings $a$ and $b$ is $ab$\\
$(a=101\hspace{10pt},b=0001 \hspace{10pt}, ab=1010001)$\\
$b$ is a substring of $s$ if $s=abc$ for some strings $a,c$ (possibly empty)\\
e.g.: (Substrings of 1001 include $\epsilon$, 001, 10, 1001)\\
A block is a maximal nonempty substring of all 0s or all 1s. \\
Example S - 0000 111 0 1 000 1 000 111     (8 blocks)\\

Main Q - How many binary string of length n satisfy certain properties? \\
Define a set S of all strings that satisfy the properties\\
Define the weight of a string to be its length\\
Answer is $$ [x^n]  \Phi_S(x)$$
Example: How many strings of length n has no 0s?\\
The set of all strings with no 0s is $S=\{ \epsilon, 1, 11, 111, 111, \hdots \}$\\
$$\Phi_S(x)=1+x+x^2+x^3+x^4+\hdots=\frac{1}{1-x}$$
Answer is $$ [x^n]  \Phi_S(x)$$
\subsection*{Two operations in regular expression of strings}
if A, B are 2 sets of strings, then the concatenation of A, B is: $$AB=\{ab|a\in A, b\in B\}$$
Example: $A=\{0,11\},\ B=\{1,11\}$, $AB=\{01, 011, 111, 1111\}$ \\
They are ``like'' cartesian products\\
Powers of strings: $A^2=AA$, $A^3=AAA$\\
Star operator $$A^*=\{\epsilon\} \cup A \cup A^2 \cup A^3 \cup \hdots=\bigcup_{k\geq 0}A^k$$
Example $\{0,1\}^*$ is all binary strings: 01101 $\in \{0,1\}^5 \subseteq \{0,1\}^*$
\NP
\section*{May 27, 2016}
How many binary strings are there of length n?\\
Generating function for $\{0,1\}$ is $2x^1$\\
Generating function for $\{0,1\}^n$ is $2^nx^n$\\
Generating function for binary strings is $$1x^0+2x^1+4x^2+\hdots+(2^nx^n)+\hdots=\frac{1}{1-2x}$$

\section*{May 30, 2016}
We saw that the binary strings with exactly k 1's decompose as $0^*(10^*)^k$, and that all binary strings decompose as $0^*(10^*)^*$\\
If $A^*$
 gives an unambiguous description of strings obtained by concatenating stings from A, then $$\Phi_{A^*}(x)=\frac{1}{1-\Phi_{A}(x)}$$
 $A=(10^*),A^*=(10^*)^*=$ al binary strings where first bit is 1 (including $\epsilon$)\\
 The non-ambiguity of the decomposition $A^*$ also implies $$\Phi_{A^k}(x)=(\Phi_{A}(x))^k$$
$A^k$ is in bijective correspondence with $A \times A \times \hdots \times A$
\end{document}