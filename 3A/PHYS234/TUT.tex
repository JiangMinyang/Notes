\documentclass[12pt, a4paper]{article}
\usepackage[letterpaper, hmargin=0.75in, vmargin=0.75in]{geometry}
\usepackage{graphicx}
\usepackage[hyphens]{url}
\usepackage{hyperref}
\usepackage{listings}
\usepackage{pgf}
\usepackage{courier}
\usepackage{amsfonts,amssymb,amsmath,amsthm,lastpage,fancyhdr,wrapfig,multirow}
\usepackage{palatino}
\usepackage{amsmath}
\usepackage[english]{babel}
\usepackage[utf8]{inputenc}
\usepackage{fancyhdr}
\textwidth 7.5in
\oddsidemargin -.5in
\topmargin -0.70in
\textheight 9.8in                      

\pagestyle{fancy}

%**************Fill in your ID and initials here*****************
\newcommand{\mc}[1]{\ensuremath{\mathcal{{#1}}}}
\newcommand{\mb}[1]{\ensuremath{\mathbb{{#1}}}}
\newcommand{\mf}[1]{\ensuremath{\mathfrak{{#1}}}}
\newcommand{\N}[1]{\ensuremath{\{1,\ldots,{#1}\}}}

\newcommand{\Worth}[1]{\{{#1} marks\}}
\newcommand{\Sln}{\smallskip \textbf{Solution.} }
\newcommand{\Extra}[1]{\{Extra credit: {#1} marks\}}


\setlength{\parskip}{0.15in}
\setlength{\parindent}{0in}


\newcommand{\NP}{\newpage \vspace*{-0.4in}}
\newcommand{\FP}{\vspace*{-0.6in}}
\newcommand{\tab}[1][1cm]{\hspace*{#1}}
\newcommand{\ES}{Erwin Schr\"odinger}

\lstset{ %
language=Java,
basicstyle=\ttfamily\scriptsize,commentstyle=\scriptsize\itshape,showstringspaces=false,breaklines=true}


\title{\huge PHYS234 TUT Notes}
\author{Minyang Jiang}
\date{\today}

\begin{document}

\maketitle
\NP
\section{Fourier Series / Fourier Transformations}
\subsection{Fourier Series}
i: complex unit\\
$\tau$: period
\begin{equation}
x(t)=\sum_{n=-\infty}^{\infty}C_ne^{\frac{i2\pi nt}{\tau}}
\end{equation}

\begin{equation}
C_n = \frac{1}{\tau}\int_{-\frac{\tau}{2}}^{\frac{\tau}{2}}x(t)e^{\frac{-i2\pi n t}{\tau}}dt
\end{equation}

\underline{Ex}:\\
\begin{enumerate}
\item
$f(x) = 1$, find the F.S.
\begin{align*}
\tau&= 2\\
f(x)& = \sum_{n=-\infty}^{\infty}C_ne^{\frac{i2\pi nx}{\tau}}\\
C_n&=\frac{1}{2}\int_{-1}^{1} 1* e^{\frac{-i2\pi n x}{2}}dx\\
&= \frac{1}{\pi n}sin(\pi n) \tab n\neq 0\\
&\int_{-1}^11dx=2 \tab n =0
\end{align*}
$$f(x) = \sum_{n=-\infty, n\neq 0}^{\infty} \frac{sin(\pi n)}{\pi n}e^{\frac{i2 \i nx}{\tau}}$$
\item 
$f(x)=1-x^2 \tab -1 \leq x\leq 1, f(x)$ period: $2$, Find F.S.
\begin{align*}
f(x)& = \sum_{n=-\infty}^{\infty}C_ne^{\frac{i2\pi nx}{\tau}} \tab \tau= 2\\
C_n&=\frac{1}{2}\int_{-1}^{1} (1-x^2) e^{\frac{-i2\pi n x}{2}}dx\\
&=\frac{1}{2}\frac{4sin(\pi n)-4\pi n cos(\pi n)}{(\pi n)^3} \tab n\neq 0\\
&\int_{-1}^1(1-x^2)dx=0 \tab n=0
\end{align*}
\end{enumerate}
\underline{Fourier Series:}
\begin{enumerate}
\item[*] need $f(x)$ to be periodical $x \in (-\infty, \infty)$
\item[*] caution about singularity of $C_n$'s
\item[*] More "flat", the messier F.S.
\item[*] follow the formula
\end{enumerate}

\subsection{Fourier Transformations}
\begin{align*}
f(\nu) &= \int_{-\infty}^{\infty}f(x)e^{-2\pi i x\nu}dx\\
f(x) &= \int_{-\infty}^{\infty}f(\nu)e^{2\pi i x \nu}d\nu \tab \text{Reverse}\\
\nu&: \text{frequency}\\
x&: \text{time}
\end{align*}
\underline{Ex:}
$f(x) = e^{-\|x\|}$
\begin{align*}
f(\nu)&=\int_{-\infty}^{\infty}e^{-\|x\|}e^{-2\pi i x \nu}dx\\
&=\int_{-\infty}^{\infty}e^{x}e^{-2\pi i x \nu}dx + \int_{-\infty}^{\infty}e^{-x}e^{-2\pi i x \nu}dx\\
&=\frac{2}{1+(e\pi \nu)^2}
\end{align*}

\section{Linear algebra}
\begin{enumerate}
\item[•] vector
\item[•] vector space
\item[•] linear combination
\item[•] linearly independence
\item[•] span
\item[•] basis
\item[•] dimension
\item[•] components of a vector
\item[•] identifying vec space
\item[•] identify comp vector
\end{enumerate}
\subsection{vector}
magmotide and directions\\
list of numbers\\
\textbf{\underline{element of a vector space}}\\
a vector is an object whihch act like a vector under transformations

\subsubsection{Vector Space}
Def: A vector space is a set V together with vector addition "+", and scalar multiplication satisfying:
\begin{enumerate}
\item associativity, $\vec{u},\vec{v},\vec{w}\in V, \vec{u}+(\vec{v}+\vec{w})=(\vec{u}+\vec{v})+\vec{w}$
\item communitativity, $\vec{u}, \vec{v} \in V, \vec{u} + \vec{v}=\vec{v}+\vec{u}$
\item existance identity of "+" $\exists \vec{0} \in v$ such that $\forall \vec{v} \in V, \vec{v}+\vec{0}=\vec{v}$
\item Existance of Inverse of "+" for each elements, $\forall \vec{v}\in V, \exists (-\vec{v})\in V$ such that $\vec{v}+(-\vec{v})=\vec{0}$
\item compatibility, $a\cdot (b\cdot \vec{v})=(ab)\cdot \vec{v}, a,b\in \mathbb{C}, \vec{v} \in V$
\item $\exists 1 \in \mathbb{C}. $ such that $1\vec{v}=\vec{v},\forall \vec{v}\in V$
\item $a(\vec{v}+\vec{u})=a\vec{u}+a\vec{v}, a \in \mathbb{C}, \vec{u}, \vec{v} \in V$
\item Distributivity, $(a+b)\vec{v}=a\vec{v}+b\vec{v}$
\end{enumerate}
Polynomials in x of Degree N-1, we call it set P
$$f(x)=a_0+a_1x+a_2x^2+\hdots+a_{N-1}x^{N-1}\tab f(x),g(x)\in P$$
$$f(x)+g(x)=(a_0+b_0)+(a_1+b_1)x+\hdots+(a_{N-1}+b_{N-1})x^{N-1}$$
$$cf(x)=ca_0+ca_1x+\hdots+ca_{N-1}x^{N-1}\tab c\in \mathbb{C}$$

V = $\{v,w\}$, $\mathbb{C}$\\
$v+w=w+v=w$\\
$v+v=v$\\
$w+w=w$\\
$a\cdot v=v \forall a\in \mathbb{C}$\\
$a\cdot w=w \forall a\in \mathbb{C}$ 


\subsubsection{linear combination}
we have $\vec{\alpha},\vec{\beta},\vec{\gamma}$, $a\vec{\alpha}+b\vec{\beta}+c\vec{\gamma}, a,b,c\in \mathbb{C}$ 

\subsubsection{linear independence}
$\vec{\gamma} is linear independent of \{\vec{\alpha}\}$ if $\vec{\gamma}$ cannot be written as a linear combination of $\vec{\alpha_i}$, $\vec{\gamma}\not = a_1\vec{\alpha_1}+\hdots+a_n\vec{\alpha_n}$

\subsubsection{Span}
$Span[\{\vec{\alpha_i}\}]=\{\sum_i \lambda_i\vec{\alpha_i}|\lambda_i\in \mathbb{C}, \vec{\alpha_i}\in \{\vec{\alpha_i}\}\}$\\
a span of $\{\vec{\alpha_i}\}$ is the set of all linear combinations of $\vec{\alpha_i}$
\subsubsection{Basis}
a basis B for a vector space V is a linearly independent set which spans V $$Span[B]=V,dim(V)=\text{number of elements in B}$$


\section{Matrics}
\begin{enumerate}
\item Linear Transformation
\item elements
\item matrix
\item sum/prods of Linear Tran
\item Row/Col matrix
\item transpose
\item main diagoal
\item symmetric metrics
\item conjugate
\item Hermition conjugate
\item adjoint
\item commutator
\item Identity
\item Inverse
\item unitary matrix
\end{enumerate}
\subsection{elements}
$T_{ij}$ are the elements of a matrix which represents the linear transformation of a particular basis
\subsection{sum}
\begin{align*}
(S+T)\vec{\phi}=s\vec{\phi}+T\vec{\phi}\\
U=S+T\\
U_{ij}=S_{ij}+T_{ij}
\end{align*}

\subsection{Product}
\begin{align*}
U=ST, T\vec{\phi}=\vec{\phi}\\
ST\vec{\phi}=S(T\phi)=S\phi
\end{align*}

\subsection{Transpose}
$$T^T, (T^T_{ij}=T_{ji})$$

\subsection{symmetric matrix}
symmetric
$$T=T^T$$
anti symmetrics
$$T=-T^T \rightarrow T_{ij}=-T{ji}, T_{11}=-T_{11}=0$$

\subsection{conjugate}
conjugate of a matrix
$$T^*$$
$$z=a+ib, z^*=a-ib$$
e.g.
$$
T=
\begin{bmatrix}
i & 6i\\
1 & 7
\end{bmatrix}
,T^*=
\begin{bmatrix}
-i & -6i\\
1 & 7
\end{bmatrix}
$$
Hermition conjugate of T 
$$(T^*)^T=T^{\dagger}$$
Hermition matrix = Adjoint Matrix
$$(T^*)^T=T,T^{\dagger}=T$$

\subsection{commutator}
$$[\hat{S},\hat{T}]=ST-TS=0$$

\subsection{Indentity Matrix}
$$\hat{I}=\delta_{ij},\hat{I}|\phi>=|\phi>, \forall	|\phi>\in V$$
$$
\delta_{ij}=
\begin{cases}
0\tab i \not = j\\
1\tab i = j
\end{cases}
$$

\subsection{Inverse}
$$T^{-1}$$
$$TT^{-1}=T^{-1}T=\hat{I}$$

\subsection{unitary matrix}
what if $$T^{\dagger}=T^{-1}$$
Then T is said to be unitary




\end{document}