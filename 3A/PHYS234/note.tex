\documentclass[12pt, a4paper]{article}
\usepackage[letterpaper, hmargin=0.75in, vmargin=0.75in]{geometry}
\usepackage{graphicx}
\usepackage[hyphens]{url}
\usepackage{hyperref}
\usepackage{listings}
\usepackage{pgf}
\usepackage{courier}
\usepackage{amsfonts,amssymb,amsmath,amsthm,lastpage,fancyhdr,wrapfig,multirow}
\usepackage{palatino}
\usepackage{amsmath}
\usepackage[english]{babel}
\usepackage[utf8]{inputenc}
\usepackage{fancyhdr}
\textwidth 7.5in
\oddsidemargin -.5in
\topmargin -0.70in
\textheight 9.8in                      

\pagestyle{fancy}

%**************Fill in your ID and initials here*****************
\newcommand{\mc}[1]{\ensuremath{\mathcal{{#1}}}}
\newcommand{\mb}[1]{\ensuremath{\mathbb{{#1}}}}
\newcommand{\mf}[1]{\ensuremath{\mathfrak{{#1}}}}
\newcommand{\N}[1]{\ensuremath{\{1,\ldots,{#1}\}}}

\newcommand{\Worth}[1]{\{{#1} marks\}}
\newcommand{\Sln}{\smallskip \textbf{Solution.} }
\newcommand{\Extra}[1]{\{Extra credit: {#1} marks\}}


\setlength{\parskip}{0.15in}
\setlength{\parindent}{0in}


\newcommand{\NP}{\newpage \vspace*{-0.4in}}
\newcommand{\FP}{\vspace*{-0.6in}}
\newcommand{\tab}[1][1cm]{\hspace*{#1}}
\newcommand{\ES}{Erwin Schr\"odinger}

\lstset{ %
language=Java,
basicstyle=\ttfamily\scriptsize,commentstyle=\scriptsize\itshape,showstringspaces=false,breaklines=true}


\title{\huge PHYS234 Notes}
\author{Minyang Jiang}
\date{\today}

\begin{document}

\maketitle

\NP
\section{History}
\paragraph{A conservative revolutionary}
In about 1908, Planck convert to the view that the quantum of action representsan irreducible phenomenon beyond the understanding of classical physics
\paragraph{Einstein in 1905}
\begin{enumerate}
\item photoelectric effect
\item dissertation, proving the existence of atoms
\item Brownian motion
\item special relativity
\item $E=mc^2$
\end{enumerate}
\paragraph{Johann Jakob Balmer's formula}
$$v=R\left(\frac{1}{n_f^2}-\frac{1}{n_i^2}\right)$$
\paragraph{Rutherford}
atom model was unstable in classical physics
\paragraph{Niels Bohr}
- grandfather of quantum physics
\begin{enumerate}
\item solve the stability problem of Rutherford's model
\item classical physics could not apply inside the atom
\item orbits have something to do with the Planck - Einstein quantum relation of the light photon ($E=hv$).
\end{enumerate}
Bohr derived Balmer's formula
\paragraph{Bohr's model of atom}
\begin{enumerate}
\item Electrons in atoms orbit the nucleus
\item Electrons can only gain and lose energy by jumping from one allowed orbit to another, absorbing or emitting EM radiation with a frequency v given by the energy gap of the levels according to the Planck relation: $$\Delta E = E_2-E_1=hv$$
angular momentum L is restricted to be an integer multiple fo a fixed unit $$L=mvr=\frac{nh}{2\pi}=nh$$ where $n=1,2,3, \hdots$ is called the principal quantum number.\\
he mixed classical and quantum physics to get $$\frac{1}{\lambda}=R\left(\frac{1}{n_f^2}-\frac{1}{n_i^2}\right)$$
\end{enumerate}
\paragraph{Important applications of QT in 20\textsuperscript{th} century}
\begin{enumerate}
\item Invention of transistors
\item Invention of lasers
\item Invention of STM
\item ...
\end{enumerate}

\section{Chapter 1 The Wave Function and The Schr\"odinger Equation}
\subsection{de Broglie's matter wave}
\begin{equation}
p=\frac{h}{\lambda}
\end{equation}
This equation is valid for electrons, ions, photons and anyother $\Rightarrow$ every particle

Paul Langevin
de Broglie's thesis is the first feeble ray of light on  the worst of our plys enigmas
\begin{align*}
L&=n\hbar \tab (n=1,2,3,\hdots)\\
L_1 &= \hbar \neq 0 \Rightarrow r_0 = 0.527 \r{A}\\
2\pi r &= n \lambda \tab (n=1,2,3,\hdots)
\end{align*}
angular momentum:
$$L=rp=\frac{n\lambda}{2\pi}*\frac{h}{r}=n\frac{h}{2\pi}=n\hbar$$

\subsection{Schr\"odinger Equation}
\begin{equation}
F=-\frac{\partial V}{\partial x} \tab v-potential
\end{equation}
Classical phys: 
\begin{align*}
F&=ma \tab (Newtonian 2nd law)\\
F&=m\frac{d^2x}{dt^2}\\
\Downarrow\\
X&=X(t)\\
\Downarrow\\
v&=\frac{dx}{dt};p=mv=m\frac{dx}{dt}\\
T&=\frac{1}{2	}mv^2=\frac{1}{2}m(\frac{dx}{dt})^2\\
E&=T+V=\frac{p^2}{2m}+V
\end{align*}
Quantum mechanics:
$$E\rightarrow i\hbar \frac{\partial}{\partial t}; p\rightarrow \c{p}=i\hbar \triangledown$$
\begin{equation}
i\hbar \frac{\partial \Psi(x,t)}{\partial t} =\left( -\frac{\hbar^2}{2m}\triangledown^2+V \right)\Psi(x,t)
\end{equation}
It is a postulate only! no proof
$$\Psi(x,t) - \text{The Wave function} \Rightarrow \text{to describe physics properties}$$

Consider free particles, from classical mechanics
\begin{equation}
E=\frac{1}{2}m\vec{v}^2=\frac{(m\vec{v}^2)}{2m}=\frac{\vec{p}^2}{2m}
\end{equation}
de Broglie's hypothesis:
\begin{align*}
E&=hv\\
\lambda&=\frac{h}{p}\\
\omega&=2\pi v\tab \text{angular frequency}\\
\| \vec{k}\|&=\frac{2\pi}{\lambda} \tab \text{wave vector}\\
\hbar&=\frac{h}{2\vec{v}}
\end{align*}
\begin{align}
E&=\hbar \omega\\
\vec{p}&=\hbar \vec{k}
\end{align}
particle motion can be described as \underline{a classical plane wave}:
\begin{align*}
\Psi(\vec{Y},t)=&\Psi_0e^{i(\vec{k}\vec{Y}-\omega t)}\\
&=\Psi_0e^{\frac{i(\vec{p}\vec{Y}-Et)}{\hbar}}\\
\frac{\partial \Psi}{\partial t}=&-\frac{iE\Psi_0}{\hbar}e^{\frac{i(\vec{p}\vec{Y}-Et)}{\hbar}}\\
\triangledown &= \frac{i\vec{p}\Psi_0}{\hbar}e^{\frac{i\vec{p}\vec{r}-Et}{\hbar}}\\
\triangledown^2 \Psi&=-\frac{\vec{p}^2}{\hbar^2}\Psi e^{\frac{i(\vec{p}\vec{Y}-Et)}{\hbar}}
\end{align*}
\begin{align}
i\hbar \frac{\partial}{\partial t}\Psi = E\Psi\\
-i\hbar \triangledown \Psi = \vec{p}\Psi\\
-\hbar^2 \triangledown^2 \Psi=\vec{p}^2\Psi
\end{align}

using Eq (4), we get
\begin{equation}
i\hbar \frac{\partial}{\partial t}\Psi=-\frac{\hbar^2 \triangledown^2}{2m}\Psi
\end{equation}
For particles in a potential $V(\vec{Y})$
\begin{align}
E&=\frac{\vec{p}^2}{2m}+V(\vec{Y})\\
i\hbar \frac{\partial \Psi(\vec{Y},t)}{\partial t}&=\left( -\frac{\hbar^2}{2m}\triangledown^2+V(\vec{Y}) \right)\Psi(\vec{Y},t)
\end{align}

\subsection{Statistical Interpretation of the wave function $\Psi(\vec{Y},t)$}
$$F=ma\rightarrow X(t) \rightarrow v=\frac{dx}{dt},p=m\frac{dx}{dt}$$
Schr\"odinger Eq $\rightarrow \Psi(\vec{Y},t)\rightarrow$ quantum state of the system $\rightarrow$ physics properties of the system \\
The wave in QM is \underline{not a wave in physical space}, it is \underline{a wave in an abastract mathematical space}\\
for free particles
\begin{align*}
E&=\frac{p^2}{2m}\Rightarrow E=\hbar \omega, p=\frac{h}{\lambda}=\hbar k\\
\omega &= \frac{\hbar k^2}{2m} \Rightarrow v_g \text{ group velosity of the wave} = \frac{d\omega}{dk}=\frac{\hbar k}{m}=\frac{p}{m}=v \text{ classical velosity}\\
\frac{d^2\omega}{dk^2}&=\frac{\hbar}{m} > 0 \Rightarrow \text{The wave packet is diverging}
\end{align*}
There is something wrong in de Broglie's hypothsis
\subsubsection{Born's Stat Interpretation}
It states that: The probability of finding a particle described by the wave function $\Psi (x,t)$ in the region, $x$ to $x+dx$ is given by $\rho(x,t)dx = \| \Psi(x,t) \|^2 dx$
$\rho(x,t)=\| \Psi(x,t) \|^2 = \Psi(x,t)^*\Psi(x,t)$

The probability of finding the particle between a and b at time t, is $$P_a^b(t)=\int_a^b\rho(x,t)dx$$

\subsubsection{physical requirements of $\Psi(x,t)$}
\begin{enumerate}
\item $\Psi$ mus be square integrable - $\int \|\Psi\|^2dx$
\item $\Psi, \frac{\partial \Psi}{\partial x}, \frac{\partial \Psi}{\partial t}$ must be finite and single-valued
\item $\Psi$ must be continuous in space
\item $\frac{\partial \Psi}{\partial x}$ is continueous except at points with potential $V=\infty$, if $\frac{\partial \Psi}{\partial x}$ is discontinuous, $\frac{\partial^2 \Psi}{\partial x^2}\rightarrow \infty$
\end{enumerate}

\subsection{Probability}
The probability of finding a particle occupying at energy level $\zeta$: $P(\zeta)= \frac{N_j}{N}$\\
Then the average energy of all the particle among the energy levels ($\zeta_1, \zeta_2,\hdots$) is: $$<\zeta_i>=\sum_{j=1}^\infty \zeta_j P(\zeta_j)$$
In QM, we are interested is to get \underline{the expectation value} that is, the average value $\not =$ the most probable value

\subsection{Normalization}
Born's Stat Interpretation $\Rightarrow$ $$\int_{-\infty}^{\infty}\rho(x,t)=\int_{-\infty}^{\infty}\|\Psi(x,t)\|^2dx=1$$
this is called normalization condition of $\Psi(x,t)$
$$i\hbar \frac{\partial \Psi}{\partial t}=\left[ -\frac{\hbar^2}{2m}\triangledown^2+V \right]\Psi$$
We can find $A\Psi$ to make normalization condition hold, if not, then it is non-normalizable, non-normalization solution cannot represent particles.

physically realizable states are represented by the square integrable solutions to Schr\"odinger Equation

If $\Psi(x,0)$ is normalizable, would $\Psi(x,t)$ be normalizable or not? \\ To prove, show: $$\frac{d}{dt}\int_{-\infty}^{\infty}\|\Psi\|^2dx=0$$
$\|\Psi\|^2dx$ is t independent

\subsection{Momontum}
Probability density $$\rho(x,t)=\|\Psi(x,t)\|^2=\Psi(x,t)*\Psi(x,t)$$
for particles ina state $\Psi(x,t)$, the expectation value of x: $$\bar x=\left<x\right>=\int_{-\infty}^{\infty}x\rho(x,t)dx$$
$$\bar V(x)=\left< V(x)\right>=\int_{-\infty}^{\infty}V(x)\rho(x,t)$$
For momentum P
$$\bar P =\left< p \right>=\int_{-\infty}^{\infty}P\rho(p,t)dp=\int_{-\infty}^{\infty}P\|\Phi(p,t)\|^2dp \tab \Phi(p,t) \text{ momentum space wave function}$$
$\Phi(p,t)$ is the Fourier transform of $\Psi(x,t)$
$$\Psi(x,t)=\frac{1}{\sqrt{2\pi\hbar}}\int_{-\infty}^{\infty}e^{\frac{ipx}{\hbar}}\Phi(p,t)dp$$
$$\Phi(x,t)=\frac{1}{\sqrt{2\pi\hbar}}\int_{-\infty}^{\infty}e^{\frac{-ipx}{\hbar}}\Psi(x,t)dx$$
$$\rho(p,t)=\|\Phi(p,t)\|^2 \tab \text{probability density in momentum space}$$
The probability of finding the particle at $p$ to $p +dp$ at time t is given by $$\|\Phi(p,t)\|^2dp$$ 
$$\int_{-\infty}^{\infty}\|\Phi(p,t)\|^2dp=\int_{-\infty}^{\infty}\|Psi(x,t)\|^2dx=1$$
\begin{align*}
\bar p=\left<p\right>=&\int_{-\infty}^{\infty}P\|\Phi(p,t)\|^2dp\\
=&\int_{-\infty}^{\infty}P\Phi^*dp\frac{1}{\sqrt{2\pi\hbar}}\int_{-\infty}^{\infty}e^{\frac{-ipx}{\hbar}}\Psi dx\\
=&\frac{1}{\sqrt{2\pi\hbar}}\int_{-\infty}^{\infty}\int_{-\infty}^{\infty}\Phi^*e^{\frac{-ipx}{\hbar}}pdp\Psi dx\\
\Psi^*=&\frac{1}{\sqrt{2\pi\hbar}}\int_{-\infty}^{\infty}e^{-\frac{ipx}{\hbar}}\Phi^*(x,t)dp\\
\frac{\partial \Psi^*}{\partial x}=&\frac{1}{\sqrt{w\pi\hbar}}(-\frac{i}{\hbar})\int_{-\infty}^{\infty}e^{-\frac{ipx}{\hbar}}p\Phi^*(p,t)dp\\
\frac{1}{\sqrt{2\pi\hbar}}\int_{-\infty}^{\infty}e^{-\frac{ipx}{\hbar}}p\Phi^*dp=&i\hbar \frac{\partial \Psi^*}{\partial x}\\
\bar p=&\left<p\right>=i\hbar \int_{-\infty}^{\infty} \frac{\partial \Psi^*}{\partial x}\Psi dx\\
=&i\hbar \int_{-\infty}^{\infty} \left[ \frac{d(\Psi^* \Psi)}{dx}-\Psi^*\frac{\partial \Psi}{\partial x} \right] dx\\
=&-i\hbar \int_{-\infty}^{\infty} \Psi^* \frac{\partial \Psi}{\partial x}dx\\
=&\int_{-\infty}^{\infty}\Psi ^*(-i\hbar \frac{\partial}{\partial	x})\Psi dx=\int_{-\infty}^{\infty}\Psi^* \hat p \Psi dx
\end{align*}
momentum operator $\hat p = -i\hbar \frac{\partial}{\partial x}=-i\hbar \triangledown$
\begin{align*}
T=\frac{1}{2}mv^2=\frac{p^2}{2m}\Rightarrow \hat T =\frac{\hat p^2}{2m}=\frac{(-i\hbar\triangledown)^2}{2m}=\frac{-\hbar^2 \triangledown ^2}{2m}\\
\hat T=\left<T \right>=\int\Psi^*\hat T\Psi dx\\
\text{Angular momentum }\hat{L}=\vec{r}\times \vec{p}\Rightarrow \hat{L}=\vec{r}\times\vec{p}=-i\hbar(\vec{r}\times\triangledown)\\
\bar L =\left< L \right>=\int_{-\infty}^{\infty}\Psi^*\hat L \Psi dx
\end{align*}
The expectation value of any dynamical quantity $Q(x,p)$
\begin{align*}
\bar Q(x,p)=\left< Q(x,p) \right>=\int_{-\infty}^{\infty}\Psi^*Q(x,-i\hbar\triangledown)\Psi dx\\
=\int_{-\infty}^{\infty}\Psi^*Q(\hat x,p)\Phi dp
\end{align*}

\section{The uncertainty principle}
Special (Fourier) analysis of a wave packet\\
From Section 1.2, the plane wave solutions:$$\Psi(x,t)=Ae^{i(kx-wt)}$$ meets the schr\"odinger Equation\\
A full Solution to the wave equation is the superposition of many $\Psi(x,t)$ with various $\omega,(\nu, \lambda, k=\frac{2\pi}{\lambda})$ \\
This is the concept of a wave packet, which can be expressed as $$\Psi(x,t)=\frac{1}{\sqrt{2\pi}}\int_{-\infty}^{\infty}\Phi(k,t)e^{i(kx-\omega t)}dk$$ $\Phi(k,t)$ is the amplitude of the wave with a wave nubmer k, which is given by Fourier transform:
$$\Phi(k,t)=\frac{1}{\sqrt{2\pi}}\int_{-\infty}^{\infty}\Psi(x,t)e^{-i(kx-\omega t)}dx$$
Example: Consider a Gaussian wave packet
$$\Psi(x)=e^{\frac{1}{2}\alpha^2 x^2} \text{ at } t = 0$$
$$\|\Psi(x)\|^2=e^{-\alpha^2 x^2}$$
$$\|\Phi(k)\|^2=\frac{1}{\alpha^2}e^{-\frac{k^2}{\alpha^2}}$$
$$\Delta x \cdot \Delta k=\frac{1}{\alpha}*\alpha=1$$
using de Broslie relation $$p=\frac{h}{\lambda}=\hbar k,\hbar=\frac{h}{2\pi} \Rightarrow \Delta P = \hbar \Delta k$$
$$\Delta x \cdot \Delta p=\Delta x \cdot \hbar \Delta k=\hbar$$
\begin{equation}
\Delta x \Delta p \geq \frac{\hbar}{2}
\end{equation}
This is the Heisenberg uncertainty principle
\begin{align*}
\delta_x \delta_p \geq \frac{\hbar}{2}\\
\delta_x=\sqrt{\left< x^2 \right> - \left< x \right> ^2}\\
\delta_p=\sqrt{\left< p^2 \right> - \left< p \right> ^2}
\end{align*}
The more  precisely determined a particles's position is, the less precisely is its mometum
\begin{align*}
\Phi(k,t)=&\int_{-\infty}^{\infty} \Psi(x,t) e^{-i(kx-\omega t)}dt\\
\Psi(x,t)=&\frac{1}{\sqrt{2\pi}}\int_{-\infty}^{\infty}\Phi(\omega, x) e^{i(kx-\omega t)}d\omega\\
\Phi(\omega, x)=&\frac{1}{\sqrt{2\pi}}\int_{-\infty}^{\infty}\Psi(x,t) e^{-i(kx-\omega t)}dt
\end{align*}

$$\Delta t\Delta\omega=1$$
$$E=h\nu=\hbar \omega,\omega = 2\pi \nu$$
$$\Delta t=\Delta E ~ \hbar$$
$$\Delta t\Delta E \geq \frac{\hbar}{2}$$

\section{Time-indepenedent Schr\"odinger Equation}
\subsection{Stationary states}
For $V=V(x,t),i\hbar \frac{\partial \Psi}{\partial t}=-\frac{\hbar^2}{2m}\frac{\partial^2\Psi}{\partial x^2}+V\Psi$
If $\Psi(x,0)$ is given, $\Psi(x,t)$ is also determined at any t\\
e.g. free particle $(E=\frac{p^2}{2m})\Rightarrow\text{plane waves}$
$$\Psi(x,0)=\frac{1}{\sqrt{w\pi\hbar}}\int_{-\infty}^{\infty}\Phi(p)e^ipx/\hbar dp$$
$$\Phi=\frac{1}{\sqrt{2\pi \hbar}}\int_{-\infty}^{\infty}\Psi(x,0)e^{-ipx/\hbar}$$
If $\Psi(x,0)$ is given $\Rightarrow \Phi(p)$
$$\Psi(x,t)=\frac{1}{\sqrt{2\pi \hbar}}\int_{-\infty}^{\infty}\Phi(p)e^{i(px-Et)/\hbar}dp\Rightarrow \Phi(x,t) \text{ at any time t}$$ 
For a special case $V(x,t)=V(x)$ independent of t, the S.E Can be solved by method of Sepervation variables
$$\Psi(x,t)=\psi(x)\phi(t)$$

the S.E becomes:
$$i\hbar \frac{d\phi}{dt}\Psi = \frac{\hbar^2}{2m}\frac{d^2\Psi}{dx^2}\phi+V\Psi \phi$$
$$\Rightarrow i\hbar \frac{1}{\phi}\frac{d\phi}{dt}=-\frac{\hbar^2}{2m}\frac{1}{\psi}\frac{d^2\psi}{dx^2}+V\text{ Seperation constant E}$$
$$\Rightarrow i\hbar \frac{1}{\phi}\frac{d\phi}{dt}=E \Rightarrow\phi=\phi_0e^{-iEt/\hbar}=e^{-iEt/\hbar} \text{ let } \phi_0=1$$
$$-\frac{\hbar^2}{2m}\frac{1}{\psi}\frac{d^2\psi}{dx^2}+V=E$$
$$\Rightarrow -\frac{\hbar^2}{2m}\frac{d^2\psi}{dx^2}+V\psi=E\psi \text{ time independent S.E.}$$

In calssical mechanics, the total energy of the particle is called the hamiltonian: $$H(x,p)=\frac{p^2}{2m}+V(x), \text{ using } \hat p=-i\hbar \triangledown=-i\hbar \frac{\partial}{\partial x}$$
$$\hat H=\frac{\hat p^2}{2m}+V(x)=-\frac{\hbar^2}{2m}\frac{\partial ^2}{\partial x^2}+V(x) \text{ -hamiltonian operator}$$
The time-independent S.E: $$\hat H \psi(x)=E\psi(x) \text{ E - eigenvalue}$$
Some interesting features:
\begin{enumerate}
\item Stationary states
$$\psi(x,t)=\psi(x)\phi(t)=\psi(x)e^{-iEt/\hbar}$$
$$\psi*(x,t)=\psi(x)e^{iEt/\hbar}$$
$$\rho(x,t)=\|\psi(x,t)\|^2=\psi^*(x,t)\psi(x,t)=\psi^*(x)\psi(x)=\rho(x)$$
For any dynamical quantity: $$\left<Q(x,p)\right>=\int \psi^* Q(x,-i\hbar\frac{\partial}{\partial x})\psi dx$$
In particular, $$\left< x\right>=\int x\|\psi(x)\|^2dx \text{ independent of t}$$
$$\left< p\right> =m \frac{d\left<x \right>}{dt}=0$$
\item They are states of definite total energy
$$\left<H \right> =\int \psi^* \hat H \psi dx=\int \psi^* E\psi dx=E\int \psi^* \psi dx = E$$
$$\left<H^2 \right>=\psi^* \hat H^2 \psi dx=\int \psi^* \hat H (\hat H \psi)dx=\int \psi^* \hat H(E\psi)dx = E^2$$
$$\sigma_E=\sigma_H=\sqrt{\left<H^2\right> -\left<H\right>^2}=0$$
The total energy = seperation constant E = well-defined (definite)
\item 
$$\Psi_1(x,t)=\Psi_1(x)e^{-iE_1t_1/\hbar}$$
$$\Psi_2(x,t)=\Psi_2(x)e^{-iE_2t_2/\hbar}$$
The general solution is a linear combination of seperable solutions:
$$\Psi(x,t)=\sum_{n=1}^{\infty}C_n\Psi_n(x,t)=\sum_{n=1}^{\infty}C_n\Psi_n(x)e^{-iE_nt/\hbar}$$
Given $V=V(x)$ and $\Psi(x,0)\Rightarrow\Psi(x,t)$ at any t
$$\Psi(x,0)=\sum_{n=1}^{\infty}C_n\Psi_n(x)\Rightarrow C_n$$
$$\Psi(x,t)=\sum_{n=1}^{\infty}C_n\Psi_n(x)e^{-iE_nt/\hbar}$$
$\Psi(x,t)$ is not stationary state 
Example 1:\\
Given $\Psi(x,0)=C_1\Psi_1(x)+C_2\Psi_2(x)$
\begin{enumerate}
\item $\Psi(x,t)=C_1\psi_1(x)e^{-iE_1t/\hbar}+\hdots$
\item $\rho(x,t)=C_1\psi_1^2+C_2\psi_2^2+2C_1C_2\psi_1\psi_2Cos[(E_2-E_1)t/\hbar]$ which is depend on time
\item $<E>=\int \Psi^*(x,t)\hat{H}\Psi(x,t)=C_1^2E_1+C_2^2E_2$
\end{enumerate}
Example2:\\
For 1 D, normalizable solutions are non-degenerate, that is each E value only corrosponds to one $\psi(x)$\\
Proof: assume $\psi_1(x)$ and $\psi_2(x)$ have the same E value
\begin{align*}
\left[ -\frac{\hbar^2}{2m}\frac{d^2}{dx^2}+V(x) \right]\psi(x)=E\psi(x)\\
\frac{d^2\psi}{dx^2}+\frac{2m}{\hbar^2}[E-V]\psi=0\\
\frac{1}{\psi}\frac{d^2\psi}{dx^2}=-\frac{2m}{\hbar^2}[E-V]\\
\frac{1}{\psi_1(x)}\frac{d^2\psi_1}{dx^2}=\frac{1}{\psi_2(x)}\frac{d^2\psi_2}{dx^2}\\
\psi_2(x)\frac{d^2 \psi_1(x)}{dx^2}=\psi_1(x)\frac{d^2 \psi_2(x)}{dx^2}\\
\frac{d}{dx}\left[ \psi_2(x)\frac{d \psi_1(x)}{dx}-\psi_1(x)\frac{d \psi_2(x)}{dx}\right]=0\\
\Rightarrow\psi_2(x)\frac{d \psi_1(x)}{dx}-\psi_1(x)\frac{d \psi_2(x)}{dx}=constant
\end{align*}
For normalizable solution $\Rightarrow \psi_1(x)\rightarrow0$ and $\psi_2(x)\rightarrow 0$ as $x\rightarrow \pm\infty$
\begin{align*}
\Rightarrow \psi_2(x)\frac{d \psi_1(x)}{dx}-\psi_1(x)\frac{d \psi_2(x)}{dx}=0\\
\frac{1}{\psi_1(x)}\frac{d\psi_1(x)}{dx}=1\frac{1}{\psi_2(x)}\frac{d\psi_2(x)}{dx}
\end{align*}
\end{enumerate}



\subsection{Infinite Square Potential Well}
$$
V(x)=
\begin{cases}
0\tab 0\leq x\leq a\\
\infty \tab x<0, x>a
\end{cases}
$$
The t-independent S.E.$$-\frac{\hbar^2}{2m}\frac{d^2\Psi}{dx^2}+0=E\Psi$$
$$\frac{d^2\Psi}{dx^2}+k^2\Psi=0, k=\frac{\sqrt{2mE}}{\hbar}$$
Linear harmonic oscillater\\
Solution: $$\Psi(x) = ASin(kx+\delta), A,\delta\rightarrow constant$$
Boundary condition $$\Psi(0)=\Psi(a)=0\Rightarrow\delta=0,sin(ka)=0\rightarrow ka=\pm n\pi,(n=1,2,3...)$$
\begin{align*}
E_n=\frac{\hbar^2k^2}{2m}=\frac{\hbar^2}{2m}\left( \frac{n\pi}{a}\right)^2=\frac{n^2\pi^2\hbar^2}{2ma^2},(n=1,2,3,\hdots)
\end{align*}


\begin{align*}
\Psi_n(x)=A_n\sin kn=A_n\sin(\frac{n\pi}{a}x)\\
\int\|\Psi_n(x)\|^2dx=1\Rightarrow\int_0^a\|A_n\|^2\sin^2(\frac{n\pi x}{a})dx=1\\
\Rightarrow\|A_n\|^2\frac{a}{2}=1\Rightarrow A=\sqrt{\frac{2}{a}}\\
\Psi_n(x)=\sqrt{\frac{2}{a}}\sin(\frac{n\pi x}{a}), (n=1,2,3...)\\
n=1 \Rightarrow \text{ ground state}\\
n>1 \Rightarrow \text{ excited state}
\end{align*}
Some important features of $\Psi_n(x)$
\begin{enumerate}
\item For n=1, $E_1=\frac{\pi^2\hbar^2}{2ma^2}>0$\\
This could be understood from the uncertainty principle.
$$\Delta x \approx a, \Delta x \Delta p \approx\hbar\Rightarrow \Delta p \approx\frac{\hbar}{a}$$
$$E=\frac{p^2}{2m}\Rightarrow \frac{\Delta p^2}{2m}\approx \frac{\hbar^2}{2ma^2}>0$$
\item
\begin{align*}
E_n&\sim n^2\\
\Delta E_n&=\frac{\pi^2\hbar^2}{2ma^2}\left[(n+1)^2 -n^2\right]=\frac{\pi^2\hbar^2}{2ma^2}(2n+1)\sim \frac{\pi^2}{\hbar^2	ma^2}n\\
\text{As } n&\rightarrow \infty, \frac{\Delta E_n}{E_n}\approx \frac{2}{n}\rightarrow 0, \text{ The energy can be considered as a continum--classical plys}
\end{align*}
\item
$\Psi_n(x)$ are muturally orthogonal
$$\int\Psi_m^*(x)\Psi_n(x)dx=\delta_{mn}=\begin{cases}1,m=n\\0,m\not =n\end{cases}$$
Proof: For $m \not = n$
\begin{align*}
\int\Psi_m^*(x)\Psi_n(x)dx=&\frac{2}{a}\int_0^a\sin(\frac{m\pi x}{a})\sin(\frac{n\pi x}{a}) dx\\
\text{using } 2\sin\alpha\sin\beta=&\cos(\alpha - \beta)-\cos(\alpha+\beta)\\
=&\frac{1}{a}\int_0^a\left[\cos(\frac{m-n}{a}x\pi)-cos(\frac{m+n}{a}x\pi) \right]dx\\
=&\left.\left[ \frac{1}{(m-n)\pi}\sin(\frac{m-n}{a}x\pi)-\frac{1}{(m+n)\pi}\sin(\frac{m+n}{a}x\pi)\right]\right\vert_0^a\\
=&\frac{1}{\pi}\left[ \frac{\sin((m-n)\pi)}{m-n}-\frac{\sin((m+n)\pi)}{m+n}\right] = 0
\end{align*}
For $m=n$ $$\int\Psi^*\Psi dx=1$$
\item $\Psi_n(x)$ are complete $\Rightarrow$ any $f(x)$ can be expressed as a linear combination of $\Psi_n(x)$:
$$f(x)=\sum_{n=1}^{\infty}C_n\Psi_n(x)=\sqrt{\frac{2}{a}}\sum_{n=1}^{\infty}C_n\sin(\frac{n\pi x}{a})$$
Seperable solutions:
\begin{align*}
\Psi_n(x,t)=&\Psi(x)e^{-iE_nt/\hbar}\\
=&\sqrt{\frac{2}{a}}\sin(\frac{n\pi x}{a})e^{-i(\frac{n^2\pi^2\hbar^2}{2ma^2})t/\hbar}
\end{align*}
The gerneral solution:
$$\Psi(x,t)=\sum_{n=1}^{\infty}C_n\Psi_n(x,t)$$
if $\Psi(x,0)$ is given, $C_n$ can be determined:
$$\Psi(x,0)=\sum_{n=1}^{\infty}C_n\Psi(x)$$
\end{enumerate}

Using Fourier Trick:
\begin{align*}
&\int\Psi_m^*\Psi(x,0)dx=\sum_{n=1}^{\infty}\int\Psi_m^*C_n\psi_n(x)\\
=&\sum_{n=1}^{\infty}C_n\int\psi_m^*\psi_n(x)dx\\
=&\sum_{n=1}^{\infty}C_n\delta (\delta_{mn}=1,\delta_{mn}=0, m\not =n)\\
=&C_m\\
\Rightarrow &C_m=\int\psi_m^*\Psi(x,0)dx, C_n=\int	\psi_n^*\Psi(x,0)dx
\end{align*}
$C_n$ is determined by $\Psi(x,0)\Rightarrow\Psi(x,t)$ at anytime.\\
The physical meaning of $C_n$:\\
$\|C_n\|^2$ is the probability taht a measurement of the total every of the particle would yield the value $E_n\Rightarrow \sum_{n=1}^{\infty}\|a\|^2=1$\\
The expectation value of the (total) energy:
$$\left< H\right>=\sum_{n=1}^{\infty}\|C_n\|^2E_n$$
Note that, both $\|C_n\|^2$ and $<H>$ are time independent\\
$\Rightarrow$conservation of enery in Q.M.

\subsection{The harmonic oscillator}
A classical oscillator: $$F=-kx=ma=m\frac{d^2x}{dt^2}=-\frac{dV}{dx}\Rightarrow V(x)=\frac{1}{2}kx^2=\frac{1}{2}m\omega^2x^2$$
angular frequency
$$\omega=\sqrt{\frac{k}{m}}$$
Hamiltonia:
\begin{align*}
H=\frac{p^2}{2m}+V=\frac{p^2}{2m}+\frac{1}{2}m\omega^2x^2\\
\hat{H}=\frac{\hat{p}^2}{2m}+\frac{1}{2}m\omega^2x^2,(\hat{p}=-i\hbar\frac{\partial}{\partial x})\\
\hat{H}\psi=E\psi=\Rightarrow \psi(x)=?
\end{align*}
\subsubsection{Algebraic method}
Let's intraduce two operators: (ladder operators)
$$a_{\pm}=\frac{1}{\sqrt{2\hbar m\omega}}(\pm i\hat{p}+m\omega x)$$
\begin{align*}
a_{+}a_{-}=&\frac{1}{\sqrt{2\hbar m \omega}}(i\hat{p} +m\omega x)(-i\hat{p}+m\omega x)\\
=&\frac{1}{2\hbar \omega m}[ \hat{p}^2 +(m\omega x)^2-im\omega (x\hat{p} -\hat{p} x)]
\end{align*}
Note that $[x\hat{p}-\hat{p} x]=[x,\hat{p}]$

\begin{align*}
a_{+}a_{-}=&\frac{1}{2\hbar \omega m}[ \hat{p}^2 +(m\omega x)^2-im\omega [x,\hat{p}]]\\
=&\frac{1}{2\hbar m \omega}[\hat{p^2}+(m\omega x)^2]-\frac{i}{2\hbar}[x,\hat{p}]\\
[x,\hat{p}]f(x)=i\hbar f(x)\Rightarrow [x,\hat{p}]=i\hbar\\
\Rightarrow a_{-}a_+=\frac{1}{\hbar \omega}[\frac{\hat{p}}{2m}+\frac{1}{2}m\omega^2x^2]-\frac{i}{2\hbar}i\hbar\\
=\frac{1}{\hbar\omega}\hat{H}+\frac{1}{2}\\
\Rightarrow \hat{H}=\hbar\omega(a_-a_+-\frac{1}{2})
\end{align*}

From $\hat{\hbar}\Psi=E\Psi$, we have
\begin{align*}
\hbar\omega(a_-a_+-\frac{1}{2})\Psi=E\Psi\\
\hbar\omega(a_-a_++\frac{1}{2})\Psi=E\Psi
\end{align*}

To prove: If $\hat{H}\Psi=E\Psi$, then 
$$\hat{H}(a_+\Psi)=(E+\hbar\omega)(a_+\Psi)$$
\begin{align*}
\hat{H}(a_+\Psi)=&\hbar\omega(a_-a_+ + \frac{1}{2})(a_+\Psi)\\
=&\hbar\omega(a_+a_-a_++\frac{1}{2}a_+)\Psi\\
=&\hbar\omega a_+(a_-a_++\frac{1}{2})\Psi\\
=&\hbar\omega a_+(a_-a_++1+\frac{1}{2})\Psi\\
=&a_+(\hbar\omega(a_-a_++\frac{1}{2})+\hbar\omega)\Psi=a_+(E\Psi+\hbar\omega\Psi)\\
=&a_+(E+\hbar\omega)\Psi\\
=&(E+\hbar\omega)(a_+\Psi)\\
\Rightarrow \hat{H}(a_+\Psi)=(E+\hbar\omega)(a_+\Psi)
\end{align*}
Similarly, we can prove:
$$\hat{H}(a_-\Psi)=(E-\hbar\omega)(a_-\Psi)$$
$\Rightarrow$ reacing the lowest energy state $\Psi_0$,$a_-\Psi_0=0$\\
$a_+$ raising operator\\
$a_-$ lowering operator\\
From $\Psi_0$,$a_-\Psi_0=0\Rightarrow \hat{p}=-i\hbar\frac{\partial}{\partial x}$
\begin{align*}
\frac{1}{\sqrt{2\hbar m \omega}}(i\hat{p}+m\omega x)\Psi_0&=0\\
(\hbar\frac{d}{dx}+m\omega x)\Psi_0=&0\\
\Rightarrow \frac{d\Psi_0}{dx}\Psi_0=-\frac{m\omega}{\hbar}xAx
\end{align*}
$$\Psi_0=Ae^{-m\omega x^2/2\hbar}$$
$$1=\int \Psi_0^*\Psi_0dx \Rightarrow A=(\frac{m\omega}{\pi\hbar}^{1/4})\Rightarrow \Psi_0=(\frac{m\omega}{\pi\hbar}^{1/4})e^{-m\omega x^2/2\hbar}$$
$$\hat{H}\Psi_0=E_0\Psi_0 \Rightarrow \hbar\omega(a_+a_-+\frac{1}{2})\Psi_0=E_0\Psi_0$$
$$a_-\Psi_0=0\rightarrow \hbar\omega * \frac{1}{2}\Psi_0=E_0\Psi_0\Rightarrow E_0=\frac{1}{2}\hbar\omega$$


$$a_+\Psi_0\Rightarrow \Psi_1, a_+\Psi_1\Rightarrow \Psi_2 \hdots$$
$$\hat{H}\Psi_1=\hat{H}(a_+\Psi_0)=(E_0+\hbar\omega)(a_+\Psi_0);\bar{H}\Psi_2=(E_0+2\hbar\omega)\Psi_2$$
\begin{align*}
\Psi_n(x)=A_n(a_+)^n\Psi_0(x)\\
E_n=(n+\frac{1}{2})\hbar\omega\\
n=1,2,3,\hdots
\end{align*}

$$A_n=\frac{1}{\sqrt{n!}}$$
Proof: 
\begin{align*}
a_+\Psi_n=C_n\Psi_{n+1};a_+\Psi_n=d_n\Psi_{n-1}
\end{align*}
Hermitian operator = for any f(x) and g(x):
$$\int_{-\infty}^{\infty}f^*(a_{\pm} g)dx=\int_{-\infty}^{\infty}(a_{\pm}f)^*gdx$$

\subsection{The Free Particle}
$$E=\frac{\vec{p}^2}{2m}, V(x)$$
$$E=hv,\lambda=\frac{h}{p}\Rightarrow\omega=2\pi v = \frac{E}{\hbar}, E = \hbar \omega, \vec{k}=\frac{\vec{p}}{\lambda}, \vec{p}=\hbar\vec{k}$$
t-depenedent S.E:
$$-\frac{\hbar^2}{2m}\frac{\partial ^2 \Psi(x,t)}{\partial x^2}=i\hbar\frac{\partial \Psi(x,t)}{\partial t}=E\Psi(x,t)$$
The Seperable solution for k's are given by $$\psi(x,t)=Ae^{i(kx-\omega t)}$$

The general solution to the t-dependent S.E is a wave packet:
$$\Psi(x,t)=\frac{1}{\sqrt{2\pi}}\int_{-\infty}^{\infty}\phi(k)e^{i(kx-\omega t)}dk$$
if $\Psi(x,0)$ is given, $\phi(k)$ will be determined by Fourier transform:
$$\Psi(x,0)=\frac{1}{\sqrt{2\pi}}\int_{-\infty}^{\infty}\psi(k)e^{ikx}dk$$
$$\phi(k)=\frac{1}{\sqrt{2\pi}}\int_{-\infty}^{\infty}\psi(x,0)e^{-ikx}dx$$
$$V_p=\frac{\omega}{k};V_g=\frac{d\omega}{dk}$$
For free particles: $$E=\hbar \omega, p=\hbar k\Rightarrow \omega = \frac{E}{\hbar}=\frac{(\hbar k)^2}{2m\hbar}=\frac{\hbar k^2}{2m}$$
$$V_p=\frac{\omega}{k}=\frac{\hbar k}{2m}=\frac{p}{2m}=\frac{m V_{classical}}{2m}=\frac{V_{classical}}{2}$$
$$V_g=\frac{d\omega}{dk}=\frac{2\hbar k}{2m}=\frac{p}{m}=V_{classical}$$


\underline{Quiz 1:}\\
Q1: planck: radiation energy is quantlized during absorption and emission\\
\tab 	E: light is composed of light quantum (photons) \\
1.$E=h\mu\Rightarrow$ total energy$=nh\mu$\\
2.$$p=\frac{h}{\lambda} \text{ photon's momentom}$$
3.$E_k^{max} = hv-\Phi$\\
\underline{Quiz 2:}\\
Q1:$\left<x\right>$: the average measurement of multiple particle \\
Q2: $p=\frac{h}{\lambda}$, limitation: one particle has one mometum and one wavelength (not true), mometum and wavelength has a distribution

The $\delta(x)$ well:\\
For bound state with $E< 0$ we got
$$
\delta(x)=
\begin{cases}
Be^{kx}\tab (x \leq 0)\\
Be^{-kx} \tab (x \geq 0)
\end{cases}
$$
$$\left.\frac{d\Psi}{dx}\right \vert_{x=0}=?$$

Integrate the t-independent S.E from $-\epsilon$ to $+\epsilon$ ($\epsilon \rightarrow 0$)
\begin{align*}
&-\frac{\hbar^2}{2m}\int_{-\epsilon}^{\epsilon}\frac{d^2\Psi}{dx^2}dx+\int_{-\epsilon}^{\epsilon}V(x)\Psi(x)dx\\
&\tab-\frac{\hbar^2}{2m}\int_{-\epsilon}^{\epsilon}\frac{d^2\Psi}{dx^2}dx = 0\\
=&\int_{-\epsilon}^{\epsilon}V(x)\Psi(x)dx\\
-\frac{-\hbar^2}{2m}[\left.\frac{d\Psi}{dx}\right\vert_{+\epsilon}-\left.\frac{d\Psi}{dx}\right\vert_{-\epsilon}]=\alpha\Psi(0)\\
\Delta(\frac{d\Psi}{dx})=-\frac{2m\alpha}{\hbar^2}\Psi(0)
\end{align*}

\begin{align*}
\left.\frac{d\Psi}{dx}\right\vert_{+\epsilon}=-BK\\
\left.\frac{d\Psi}{dx}\right\vert_{-\epsilon}=BK
-BK-BK=-\frac{2m\alpha}{\hbar^2}\Psi(0)=-\frac{2m\alpha}{\hbar^2}B\\
\Rightarrow k = \frac{m\alpha}{\hbar^2}\\
k=\frac{\sqrt{-2mE}}{\hbar}\Rightarrow E = -\frac{\hbar^2 k^2}{2m} = -\frac{m\alpha^2}{2\hbar^2}\\
\int_{-\infty}^{\infty}|\Psi(x)|^2dx=2|B|^2\\
\int_{-\infty}^{\infty}e^{-2kx}dx = \frac{|B|^2}{k}=1
\end{align*}
For scattering states with $E>0$, For $x<0$ the S.E gives
$$\frac{d^2\Psi}{dx^2}=-\frac{2mE}{\hbar^2}\Psi=-k^2\Psi,k=\frac{\sqrt{2mE}}{\hbar}$$
The general solution:
$$\Psi(x) = Ae^{ikx}+Be^{-ikx}$$
For $x>0$, $\Psi(x)=Fe^{ikx}+Ge^{-ikx}$
\begin{enumerate}
\item $\Psi(x)$ is continuous at $x=0\Rightarrow A+b=F+G$
\begin{align*}
\left.\frac{d\Psi}{dx}\right\vert_{-G}=ik(Ae^{ikx}-Be^{-ikx})\vert_{x=0}=ik(A-B)\\
\left.\frac{d\Psi}{dx}\right\vert_{+G}=ik(F-G)\\
\Psi(0)=A+B=F+G\\
\Rightarrow ik(F-G-A+B)=-\frac{2m\alpha}{\hbar^2}(A+B)
\end{align*}
\begin{align*}
F-G=&A-B+\frac{2m\alpha}{\hbar^2k}i(A+B)=A-B+2i\beta(A+B)\tab (\beta = \frac{m\alpha}{\hbar^2 k})\\
F=&A(1+2i\beta)-B(1-2i\beta)
\end{align*}
\begin{align*}
G=0\Rightarrow A+B=F=A(1+2i\beta)-B(1-2i\beta)\\
\Rightarrow B=\frac{i\beta}{1-i\beta}A; F=\frac{1}{1-i\beta}A
\end{align*}
The transmission coefficient T, that is, the probability of an incredent particlee being transimitted througth the potential well 
$$T=\frac{|F|^2}{|A|^2}=\frac{1}{\beta^2}\tab (\beta = \frac{m\alpha}{\hbar^2 k})$$
The reflection coefficient $$R=\frac{|B|^2}{|A|^2}=\frac{\beta^2}{1+\beta^2}=1-T$$
\end{enumerate}


\section{}
\subsection{Operators and ovservables}
The conjugate of an operator $\hat{Q}$ is $\hat{Q}^*$\\
The hermitian conjugate of an operator $\hat{Q}$ is called the operator $\hat{Q}^{\dagger}$
$$\left<f|\hat{Q}g\right>=\left<\hat{Q}^{\dagger}f|g\right>$$
or
$$\int f^* \hat{Q}gdx=\int (\hat{Q}^{\dagger}f)^*gdx$$
\begin{align*}
\left<f|\hat{Q}g\right>=\int f^*xgdx=\int (xf)^*gdx=\left<xf|g\right>\\
\Rightarrow \hat{x}^{\dagger}=x
\end{align*}
\begin{align*}
\left<f|\hat{p}g\right>=\int f^*\hat{p}gdx=\int f^*(-i\hbar\frac{d}{dx})gdx=-i\hbar\int f^*\frac{dg}{dx}dx=-i\hbar f^*g+i\hbar\int \frac{df^*}{dx}gdx\\
=i\hbar \int \frac{df^*}{dx}gdx=\int (-i\hbar \frac{d}{dx}f)^*gdx=\left<\hat{p}^{\dagger}f|g\right>\\
\Rightarrow\hat{p}^{\dagger}=-i\hbar \frac{d}{dx}=\hat{p}
\end{align*}
similarly, $\hat{a_+}^{\dagger}=\hat{a_-}$\\
Generally, it can be proven that $(\hat{A}\hat{B})^{\dagger}=\hat{B}^{\dagger}\hat{A}^{\dagger}$\\
If $\hat{Q}^{\dagger}=\hat{Q}$ then $\hat{Q}$ is called a hermitian operator\\
$$\left<f|\hat{Q}g\right>=\left<\hat{Q}^{\dagger}f|g\right>=\left<\hat{Q}f|g\right>$$
$$\int f^*\hat{Q}gdx=\int (\hat{Q}f)^*gdx$$
e.g. $\hat{x},\hat{p},\hat{H}$ are hermition operators, whereas $\hat{a_+},\hat{a_-},i\hbar,\frac{d}{dx}$ are not hermitian operators\\
The sum of two hermitian operators is also a hermitian operator:
$$\hat{A}+\hat{B}=\hat{C}\text{ if }\hat{A},\hat{B}\text{ are hermitian} \Rightarrow C-hermitian$$
$\hat{A}\hat{B}$ is not a hermitian unless $[\hat{A},\hat{B}]=0$\\
Theorem 1: the expectation value of an hermitian operator must be real:\\
Proof: \begin{align*}
\hat{Q}=\int \Psi^*\hat{Q}\Psi dx=\left<\Psi|\hat{Q}\Psi\right>\\
=\left<\hat{Q}\Psi|\Psi\right>=\left<\Psi|\hat{Q}\Psi\right>^*=\left<\hat{Q}\right>^*\\
\Rightarrow \left<\hat{Q}\right> \text{ must be a real number}
\end{align*}
Theorem 2: An operator that has a real expectation value in any satte must be a hermition operator\\
Proof: $\left<\hat{Q}\right>=\left<\hat{Q}\right>^*$ in any state (function) $\Psi(x)$, $$\left<\Psi|\hat{Q}\Psi\right>=\left<\Psi|\hat{Q}\Psi\right>^*=\left<\hat{Q}\Psi|\Psi\right>$$ Let $\Psi(x)=f(x)=cg(x)$
\begin{align*}
\left<\Psi|\hat{Q}\Psi\right>=&\left<(f+cg)|\hat{Q}(f+cg)\right>\\
=&\left<f|\hat{Q}f\right>+c\left<f|\hat{Q}g\right>+c^*\left<g|\hat{Q}f\right>+|c|^2\left<g|\hat{Q}g\right>\\
\left<\hat{Q}\Psi|\Psi\right>=&\left<\hat{Q}f|f\right>+c\left<\hat{Q}f|g\right>+c^*\left<Qg|f\right>+|c|^2\left<g|\hat{Q}g\right>\\
\left<f|\hat{Q}f\right>=&\left<\hat{Q}f|f\right> \text{ and } \left<g|\hat{Q}g\right>=\left<\hat{Q}g|g\right>\\
\end{align*}
\begin{align*}
c\left<f|\hat{Q}g\right>+c^*\left<g|\hat{Q}f\right>=c\left<\hat{Q}f|g\right>+c^*\left<Qg|f\right>\\
\text{Let } c=1\Rightarrow \left<f|\hat{Q}g\right>+\left<g|\hat{Q}f\right>=\left<Qf|g\right>+\left<\hat{Q}g|f\right>\\
\text{Let } c=i\Rightarrow \left<f|\hat{Q}g\right>-\left<g|\hat{Q}f\right>=\left<Qf|g\right>-\left<\hat{Q}g|f\right>\\
\Rightarrow \left<f|\hat{Q}g\right>=\left<\hat{Q}f|g\right>\Rightarrow \hat{Q}=\hat{Q}^{\dagger}\Rightarrow \text{ Q is a harmitian operator}
\end{align*}
It follows that operators represents any physical observables must be hermitian operators
\subsection{Eigenvalues and eigen function of a hermitian operator}
In ordinary cases, when you measure an observable (physical quantity) Q in a quantum system/state $\Psi$, you would get several or many possible results ----- indeterminacy in Q.M.\\
In chapter 2, however, the total energy in a stationary state $\Rightarrow E_n,\hat{H}\Psi_n=E_n\Psi_n,\sigma_E=0$ 
\subsection{Continuous Spectrum}
The eigenvalues are continuous and the eigenfunction are non-normalizable;
Find the eigenfunction and eigenvalue of the position operator $\hat{x}$
$$x\Psi_s(x)=S\Psi_s(x)$$
\begin{align*}
&\Psi_s(x)=c\delta(x-s)\\
&\left<\Psi_{s'}|\Psi_s\right>=\int\Psi_{s'}^*|\Psi_sdx=|c|^2\int\delta(x-s')\delta(x-s)dx=|c|^2\delta(s-s')
&\text{let }|c|^2=1\\
&\left<\Psi_{s'}|\Psi_s\right>=\delta(s-s')
\end{align*}

\end{document}