\documentclass[12pt, a4paper]{article}
\usepackage[letterpaper, hmargin=0.75in, vmargin=0.75in]{geometry}
\usepackage{graphicx}
\usepackage[hyphens]{url}
\usepackage{hyperref}
\usepackage{listings}
\usepackage{pgf}
\usepackage{courier}
\usepackage{amsfonts,amssymb,amsmath,amsthm,lastpage,fancyhdr,wrapfig,multirow}
\usepackage{palatino}
\usepackage{amsmath}
\usepackage[english]{babel}
\usepackage[utf8]{inputenc}
\usepackage{fancyhdr}
\textwidth 7.5in
\oddsidemargin -.5in
\topmargin -0.70in
\textheight 9.8in                      

\pagestyle{fancy}

%**************Fill in your ID and initials here*****************
\newcommand{\mc}[1]{\ensuremath{\mathcal{{#1}}}}
\newcommand{\mb}[1]{\ensuremath{\mathbb{{#1}}}}
\newcommand{\mf}[1]{\ensuremath{\mathfrak{{#1}}}}
\newcommand{\N}[1]{\ensuremath{\{1,\ldots,{#1}\}}}

\newcommand{\Worth}[1]{\{{#1} marks\}}
\newcommand{\Sln}{\smallskip \textbf{Solution.} }
\newcommand{\Extra}[1]{\{Extra credit: {#1} marks\}}


\setlength{\parskip}{0.15in}
\setlength{\parindent}{0in}


\newcommand{\NP}{\newpage \vspace*{-0.4in}}
\newcommand{\FP}{\vspace*{-0.6in}}
\newcommand{\tab}[1][1cm]{\hspace*{#1}}
\newcommand{\ES}{Erwin Schr\"odinger}

\lstset{ %
language=Java,
basicstyle=\ttfamily\scriptsize,commentstyle=\scriptsize\itshape,showstringspaces=false,breaklines=true}


\title{\huge PHYS234 Notes}
\author{Minyang Jiang}
\date{\today}

\begin{document}

\maketitle

\NP
\section{History}
\paragraph{A conservative revolutionary}
In about 1908, Planck convert to the view that the quantum of action representsan irreducible phenomenon beyond the understanding of classical physics
\paragraph{Einstein in 1905}
\begin{enumerate}
\item photoelectric effect
\item dissertation, proving the existence of atoms
\item Brownian motion
\item special relativity
\item $E=mc^2$
\end{enumerate}
\paragraph{Johann Jakob Balmer's formula}
$$v=R\left(\frac{1}{n_f^2}-\frac{1}{n_i^2}\right)$$
\paragraph{Rutherford}
atom model was unstable in classical physics
\paragraph{Niels Bohr}
- grandfather of quantum physics
\begin{enumerate}
\item solve the stability problem of Rutherford's model
\item classical physics could not apply inside the atom
\item orbits have something to do with the Planck - Einstein quantum relation of the light photon ($E=hv$).
\end{enumerate}
Bohr derived Balmer's formula
\paragraph{Bohr's model of atom}
\begin{enumerate}
\item Electrons in atoms orbit the nucleus
\item Electrons can only gain and lose energy by jumping from one allowed orbit to another, absorbing or emitting EM radiation with a frequency v given by the energy gap of the levels according to the Planck relation: $$\Delta E = E_2-E_1=hv$$
angular momentum L is restricted to be an integer multiple fo a fixed unit $$L=mvr=\frac{nh}{2\pi}=nh$$ where $n=1,2,3, \hdots$ is called the principal quantum number.\\
he mixed classical and quantum physics to get $$\frac{1}{\lambda}=R\left(\frac{1}{n_f^2}-\frac{1}{n_i^2}\right)$$
\end{enumerate}
\paragraph{Important applications of QT in 20\textsuperscript{th} century}
\begin{enumerate}
\item Invention of transistors
\item Invention of lasers
\item Invention of STM
\item ...
\end{enumerate}

\section{Chapter 1 The Wave Function and The Schr\"odinger Equation}
\subsection{de Broglie's matter wave}
\begin{equation}
p=\frac{h}{\lambda}
\end{equation}
This equation is valid for electrons, ions, photons and anyother $\Rightarrow$ every particle

Paul Langevin
de Broglie's thesis is the first feeble ray of light on  the worst of our plys enigmas
\begin{align*}
L&=n\hbar \tab (n=1,2,3,\hdots)\\
L_1 &= \hbar \neq 0 \Rightarrow r_0 = 0.527 \r{A}\\
2\pi r &= n \lambda \tab (n=1,2,3,\hdots)
\end{align*}
angular momentum:
$$L=rp=\frac{n\lambda}{2\pi}*\frac{h}{r}=n\frac{h}{2\pi}=n\hbar$$

\subsection{Schr\"odinger Equation}
\begin{equation}
F=-\frac{\partial V}{\partial x} \tab v-potential
\end{equation}
Classical phys: 
\begin{align*}
F&=ma \tab (Newtonian 2nd law)\\
F&=m\frac{d^2x}{dt^2}\\
\Downarrow\\
X&=X(t)\\
\Downarrow\\
v&=\frac{dx}{dt};p=mv=m\frac{dx}{dt}\\
T&=\frac{1}{2	}mv^2=\frac{1}{2}m(\frac{dx}{dt})^2\\
E&=T+V=\frac{p^2}{2m}+V
\end{align*}
Quantum mechanics:
$$E\rightarrow i\hbar \frac{\partial}{\partial t}; p\rightarrow \c{p}=i\hbar \triangledown$$
\begin{equation}
i\hbar \frac{\partial \Psi(x,t)}{\partial t} =\left( -\frac{\hbar^2}{2m}\triangledown^2+V \right)\Psi(x,t)
\end{equation}
It is a postulate only! no proof
$$\Psi(x,t) - \text{The Wave function} \Rightarrow \text{to describe physics properties}$$

Consider free particles, from classical mechanics
\begin{equation}
E=\frac{1}{2}m\vec{v}^2=\frac{(m\vec{v}^2)}{2m}=\frac{\vec{p}^2}{2m}
\end{equation}
de Broglie's hypothesis:
\begin{align*}
E&=hv\\
\lambda&=\frac{h}{p}\\
\omega&=2\pi v\tab \text{angular frequency}\\
\| \vec{k}\|&=\frac{2\pi}{\lambda} \tab \text{wave vector}\\
\hbar&=\frac{h}{2\vec{v}}
\end{align*}
\begin{align}
E&=\hbar \omega\\
\vec{p}&=\hbar \vec{k}
\end{align}
particle motion can be described as \underline{a classical plane wave}:
\begin{align*}
\Psi(\vec{Y},t)=&\Psi_0e^{i(\vec{k}\vec{Y}-\omega t)}\\
&=\Psi_0e^{\frac{i(\vec{p}\vec{Y}-Et)}{\hbar}}\\
\frac{\partial \Psi}{\partial t}=&-\frac{iE\Psi_0}{\hbar}e^{\frac{i(\vec{p}\vec{Y}-Et)}{\hbar}}\\
\triangledown &= \frac{i\vec{p}\Psi_0}{\hbar}e^{\frac{i\vec{p}\vec{r}-Et}{\hbar}}\\
\triangledown^2 \Psi&=-\frac{\vec{p}^2}{\hbar^2}\Psi e^{\frac{i(\vec{p}\vec{Y}-Et)}{\hbar}}
\end{align*}
\begin{align}
i\hbar \frac{\partial}{\partial t}\Psi = E\Psi\\
-i\hbar \triangledown \Psi = \vec{p}\Psi\\
-\hbar^2 \triangledown^2 \Psi=\vec{p}^2\Psi
\end{align}

using Eq (4), we get
\begin{equation}
i\hbar \frac{\partial}{\partial t}\Psi=-\frac{\hbar^2 \triangledown^2}{2m}\Psi
\end{equation}
For particles in a potential $V(\vec{Y})$
\begin{align}
E&=\frac{\vec{p}^2}{2m}+V(\vec{Y})\\
i\hbar \frac{\partial \Psi(\vec{Y},t)}{\partial t}&=\left( -\frac{\hbar^2}{2m}\triangledown^2+V(\vec{Y}) \right)\Psi(\vec{Y},t)
\end{align}

\subsection{Statistical Interpretation of the wave function $\Psi(\vec{Y},t)$}
$$F=ma\rightarrow X(t) \rightarrow v=\frac{dx}{dt},p=m\frac{dx}{dt}$$
Schr\"odinger Eq $\rightarrow \Psi(\vec{Y},t)\rightarrow$ quantum state of the system $\rightarrow$ physics properties of the system \\
The wave in QM is \underline{not a wave in physical space}, it is \underline{a wave in an abastract mathematical space}\\
for free particles
\begin{align*}
E&=\frac{p^2}{2m}\Rightarrow E=\hbar \omega, p=\frac{h}{\lambda}=\hbar k\\
\omega &= \frac{\hbar k^2}{2m} \Rightarrow v_g \text{ group velosity of the wave} = \frac{d\omega}{dk}=\frac{\hbar k}{m}=\frac{p}{m}=v \text{ classical velosity}\\
\frac{d^2\omega}{dk^2}&=\frac{\hbar}{m} > 0 \Rightarrow \text{The wave packet is diverging}
\end{align*}
There is something wrong in de Broglie's hypothsis
\subsubsection{Born's Stat Interpretation}
It states that: The probability of finding a particle described by the wave function $\Psi (x,t)$ in the region, $x$ to $x+dx$ is given by $\rho(x,t)dx = \| \Psi(x,t) \|^2 dx$
$\rho(x,t)=\| \Psi(x,t) \|^2 = \Psi(x,t)^*\Psi(x,t)$

The probability of finding the particle between a and b at time t, is $$P_a^b(t)=\int_a^b\rho(x,t)dx$$

\subsubsection{physical requirements of $\Psi(x,t)$}
\begin{enumerate}
\item $\Psi$ mus be square integrable - $\int \|\Psi\|^2dx$
\item $\Psi, \frac{\partial \Psi}{\partial x}, \frac{\partial \Psi}{\partial t}$ must be finite and single-valued
\item $\Psi$ must be continuous in space
\item $\frac{\partial \Psi}{\partial x}$ is continueous except at points with potential $V=\infty$, if $\frac{\partial \Psi}{\partial x}$ is discontinuous, $\frac{\partial^2 \Psi}{\partial x^2}\rightarrow \infty$
\end{enumerate}

\subsection{Probability}
The probability of finding a particle occupying at energy level $\zeta$: $P(\zeta)= \frac{N_j}{N}$\\
Then the average energy of all the particle among the energy levels ($\zeta_1, \zeta_2,\hdots$) is: $$<\zeta_i>=\sum_{j=1}^\infty \zeta_j P(\zeta_j)$$
In QM, we are interested is to get \underline{the expectation value} that is, the average value $\not =$ the most probable value

\subsection{Normalization}
Born's Stat Interpretation $\Rightarrow$ $$\int_{-\infty}^{\infty}\rho(x,t)=\int_{-\infty}^{\infty}\|\Psi(x,t)\|^2dx=1$$
this is called normalization condition of $\Psi(x,t)$
$$i\hbar \frac{\partial \Psi}{\partial t}=\left[ -\frac{\hbar^2}{2m}\triangledown^2+V \right]\Psi$$
We can find $A\Psi$ to make normalization condition hold, if not, then it is non-normalizable, non-normalization solution cannot represent particles.

physically realizable states are represented by the square integrable solutions to Schr\"odinger Equation

If $\Psi(x,0)$ is normalizable, would $\Psi(x,t)$ be normalizable or not? \\ To prove, show: $$\frac{d}{dt}\int_{-\infty}^{\infty}\|\Psi\|^2dx=0$$
$\|\Psi\|^2dx$ is t independent

\subsection{Momontum}
Probability density $$\rho(x,t)=\|\Psi(x,t)\|^2=\Psi(x,t)*\Psi(x,t)$$
for particles ina state $\Psi(x,t)$, the expectation value of x: $$\bar x=\left<x\right>=\int_{-\infty}^{\infty}x\rho(x,t)dx$$
$$\bar V(x)=\left< V(x)\right>=\int_{-\infty}^{\infty}V(x)\rho(x,t)$$
For momentum P
$$\bar P =\left< p \right>=\int_{-\infty}^{\infty}P\rho(p,t)dp=\int_{-\infty}^{\infty}P\|\Phi(p,t)\|^2dp \tab \Phi(p,t) \text{ momentum space wave function}$$
$\Phi(p,t)$ is the Fourier transform of $\Psi(x,t)$
$$\Psi(x,t)=\frac{1}{\sqrt{2\pi\hbar}}\int_{-\infty}^{\infty}e^{\frac{ipx}{\hbar}}\Phi(p,t)dp$$
$$\Phi(x,t)=\frac{1}{\sqrt{2\pi\hbar}}\int_{-\infty}^{\infty}e^{\frac{-ipx}{\hbar}}\Psi(x,t)dx$$
$$\rho(p,t)=\|\Phi(p,t)\|^2 \tab \text{probability density in momentum space}$$
The probability of finding the particle at $p$ to $p +dp$ at time t is given by $$\|\Phi(p,t)\|^2dp$$ 
$$\int_{-\infty}^{\infty}\|\Phi(p,t)\|^2dp=\int_{-\infty}^{\infty}\|Psi(x,t)\|^2dx=1$$
\begin{align*}
\bar p=\left<p\right>=&\int_{-\infty}^{\infty}P\|\Phi(p,t)\|^2dp\\
=&\int_{-\infty}^{\infty}P\Phi^*dp\frac{1}{\sqrt{2\pi\hbar}}\int_{-\infty}^{\infty}e^{\frac{-ipx}{\hbar}}\Psi dx\\
=&\frac{1}{\sqrt{2\pi\hbar}}\int_{-\infty}^{\infty}\int_{-\infty}^{\infty}\Phi^*e^{\frac{-ipx}{\hbar}}pdp\Psi dx\\
\Psi^*=&\frac{1}{\sqrt{2\pi\hbar}}\int_{-\infty}^{\infty}e^{-\frac{ipx}{\hbar}}\Phi^*(x,t)dp\\
\frac{\partial \Psi^*}{\partial x}=&\frac{1}{\sqrt{w\pi\hbar}}(-\frac{i}{\hbar})\int_{-\infty}^{\infty}e^{-\frac{ipx}{\hbar}}p\Phi^*(p,t)dp\\
\frac{1}{\sqrt{2\pi\hbar}}\int_{-\infty}^{\infty}e^{-\frac{ipx}{\hbar}}p\Phi^*dp=&i\hbar \frac{\partial \Psi^*}{\partial x}\\
\bar p=&\left<p\right>=i\hbar \int_{-\infty}^{\infty} \frac{\partial \Psi^*}{\partial x}\Psi dx\\
=&i\hbar \int_{-\infty}^{\infty} \left[ \frac{d(\Psi^* \Psi)}{dx}-\Psi^*\frac{\partial \Psi}{\partial x} \right] dx\\
=&-i\hbar \int_{-\infty}^{\infty} \Psi^* \frac{\partial \Psi}{\partial x}dx\\
=&\int_{-\infty}^{\infty}\Psi ^*(-i\hbar \frac{\partial}{\partial	x})\Psi dx=\int_{-\infty}^{\infty}\Psi^* \hat p \Psi dx
\end{align*}
momentum operator $\hat p = -i\hbar \frac{\partial}{\partial x}=-i\hbar \triangledown$
\begin{align*}
T=\frac{1}{2}mv^2=\frac{p^2}{2m}\Rightarrow \hat T =\frac{\hat p^2}{2m}=\frac{(-i\hbar\triangledown)^2}{2m}=\frac{-\hbar^2 \triangledown ^2}{2m}\\
\hat T=\left<T \right>=\int\Psi^*\hat T\Psi dx\\
\text{Angular momentum }\hat{L}=\vec{r}\times \vec{p}\Rightarrow \hat{L}=\vec{r}\times\vec{p}=-i\hbar(\vec{r}\times\triangledown)\\
\bar L =\left< L \right>=\int_{-\infty}^{\infty}\Psi^*\hat L \Psi dx
\end{align*}
The expectation value of any dynamical quantity $Q(x,p)$
\begin{align*}
\bar Q(x,p)=\left< Q(x,p) \right>=\int_{-\infty}^{\infty}\Psi^*Q(x,-i\hbar\triangledown)\Psi dx\\
=\int_{-\infty}^{\infty}\Psi^*Q(\hat x,p)\Phi dp
\end{align*}

\section{The uncertainty principle}
Special (Fourier) analysis of a wave packet\\
From Section 1.2, the plane wave solutions:$$\Psi(x,t)=Ae^{i(kx-wt)}$$ meets the schr\"odinger Equation\\
A full Solution to the wave equation is the superposition of many $\Psi(x,t)$ with various $\omega,(\nu, \lambda, k=\frac{2\pi}{\lambda})$ \\
This is the concept of a wave packet, which can be expressed as $$\Psi(x,t)=\frac{1}{\sqrt{2\pi}}\int_{-\infty}^{\infty}\Phi(k,t)e^{i(kx-\omega t)}dk$$ $\Phi(k,t)$ is the amplitude of the wave with a wave nubmer k, which is given by Fourier transform:
$$\Phi(k,t)=\frac{1}{\sqrt{2\pi}}\int_{-\infty}^{\infty}\Psi(x,t)e^{-i(kx-\omega t)}dx$$
Example: Consider a Gaussian wave packet
$$\Psi(x)=e^{\frac{1}{2}\alpha^2 x^2} \text{ at } t = 0$$
$$\|\Psi(x)\|^2=e^{-\alpha^2 x^2}$$
$$\|\Phi(k)\|^2=\frac{1}{\alpha^2}e^{-\frac{k^2}{\alpha^2}}$$
$$\Delta x \cdot \Delta k=\frac{1}{\alpha}*\alpha=1$$
using de Broslie relation $$p=\frac{h}{\lambda}=\hbar k,\hbar=\frac{h}{2\pi} \Rightarrow \Delta P = \hbar \Delta k$$
$$\Delta x \cdot \Delta p=\Delta x \cdot \hbar \Delta k=\hbar$$
\begin{equation}
\Delta x \Delta p \geq \frac{\hbar}{2}
\end{equation}
This is the Heisenberg uncertainty principle
\begin{align*}
\delta_x \delta_p \geq \frac{\hbar}{2}\\
\delta_x=\sqrt{\left< x^2 \right> - \left< x \right> ^2}\\
\delta_p=\sqrt{\left< p^2 \right> - \left< p \right> ^2}
\end{align*}
The more  precisely determined a particles's position is, the less precisely is its mometum
\begin{align*}
\Phi(k,t)=&\int_{-\infty}^{\infty} \Psi(x,t) e^{-i(kx-\omega t)}dt\\
\Psi(x,t)=&\frac{1}{\sqrt{2\pi}}\int_{-\infty}^{\infty}\Phi(\omega, x) e^{i(kx-\omega t)}d\omega\\
\Phi(\omega, x)=&\frac{1}{\sqrt{2\pi}}\int_{-\infty}^{\infty}\Psi(x,t) e^{-i(kx-\omega t)}dt
\end{align*}

$$\Delta t\Delta\omega=1$$
$$E=h\nu=\hbar \omega,\omega = 2\pi \nu$$
$$\Delta t=\Delta E ~ \hbar$$
$$\Delta t\Delta E \geq \frac{\hbar}{2}$$

\section{Time-indepenedent Schr\"odinger Equation}
\subsection{Stationary states}
For $V=V(x,t),i\hbar \frac{\partial \Psi}{\partial t}=-\frac{\hbar^2}{2m}\frac{\partial^2\Psi}{\partial x^2}+V\Psi$
If $\Psi(x,0)$ is given, $\Psi(x,t)$ is also determined at any t\\
e.g. free particle $(E=\frac{p^2}{2m})\Rightarrow\text{plane waves}$
$$\Psi(x,0)=\frac{1}{\sqrt{w\pi\hbar}}\int_{-\infty}^{\infty}\Phi(p)e^ipx/\hbar dp$$
$$\Phi=\frac{1}{\sqrt{2\pi \hbar}}\int_{-\infty}^{\infty}\Psi(x,0)e^{-ipx/\hbar}$$
If $\Psi(x,0)$ is given $\Rightarrow \Phi(p)$
$$\Psi(x,t)=\frac{1}{\sqrt{2\pi \hbar}}\int_{-\infty}^{\infty}\Phi(p)e^{i(px-Et)/\hbar}dp\Rightarrow \Phi(x,t) \text{ at any time t}$$ 
For a special case $V(x,t)=V(x)$ independent of t, the S.E Can be solved by method of Sepervation variables
$$\Psi(x,t)=\psi(x)\phi(t)$$

the S.E becomes:
$$i\hbar \frac{d\phi}{dt}\Psi = \frac{\hbar^2}{2m}\frac{d^2\Psi}{dx^2}\phi+V\Psi \phi$$
$$\Rightarrow i\hbar \frac{1}{\phi}\frac{d\phi}{dt}=-\frac{\hbar^2}{2m}\frac{1}{\psi}\frac{d^2\psi}{dx^2}+V\text{ Seperation constant E}$$
$$\Rightarrow i\hbar \frac{1}{\phi}\frac{d\phi}{dt}=E \Rightarrow\phi=\phi_0e^{-iEt/\hbar}=e^{-iEt/\hbar} \text{ let } \phi_0=1$$
$$-\frac{\hbar^2}{2m}\frac{1}{\psi}\frac{d^2\psi}{dx^2}+V=E$$
$$\Rightarrow -\frac{\hbar^2}{2m}\frac{d^2\psi}{dx^2}+V\psi=E\psi \text{ time independent S.E.}$$

In calssical mechanics, the total energy of the particle is called the hamiltonian: $$H(x,p)=\frac{p^2}{2m}+V(x), \text{ using } \hat p=-i\hbar \triangledown=-i\hbar \frac{\partial}{\partial x}$$
$$\hat H=\frac{\hat p^2}{2m}+V(x)=-\frac{\hbar^2}{2m}\frac{\partial ^2}{\partial x^2}+V(x) \text{ -hamiltonian operator}$$
The time-independent S.E: $$\hat H \psi(x)=E\psi(x) \text{ E - eigenvalue}$$
Some interesting features:
\begin{enumerate}
\item Stationary states
$$\psi(x,t)=\psi(x)\phi(t)=\psi(x)e^{-iEt/\hbar}$$
$$\psi*(x,t)=\psi(x)e^{iEt/\hbar}$$
$$\rho(x,t)=\|\psi(x,t)\|^2=\psi^*(x,t)\psi(x,t)=\psi^*(x)\psi(x)=\rho(x)$$
For any dynamical quantity: $$\left<Q(x,p)\right>=\int \psi^* Q(x,-i\hbar\frac{\partial}{\partial x})\psi dx$$
In particular, $$\left< x\right>=\int x\|\psi(x)\|^2dx \text{ independent of t}$$
$$\left< p\right> =m \frac{d\left<x \right>}{dt}=0$$
\item They are states of definite total energy
$$\left<H \right> =\int \psi^* \hat H \psi dx=\int \psi^* E\psi dx=E\int \psi^* \psi dx = E$$
$$\left<H^2 \right>=\psi^* \hat H^2 \psi dx=\int \psi^* \hat H (\hat H \psi)dx=\int \psi^* \hat H(E\psi)dx = E^2$$
$$\sigma_E=\sigma_H=\sqrt{\left<H^2\right> -\left<H\right>^2}=0$$
The total energy = seperation constant E = well-defined (definite)
\end{enumerate}


















































\end{document}