\documentclass[12pt, a4paper]{article}
\usepackage[letterpaper, hmargin=0.75in, vmargin=0.75in]{geometry}
\usepackage{graphicx}
\usepackage[hyphens]{url}
\usepackage{hyperref}
\usepackage{listings}
\usepackage{pgf}
\usepackage{courier}
\usepackage{amsfonts,amssymb,amsmath,amsthm,lastpage,fancyhdr,wrapfig,multirow}
\usepackage{palatino}
\usepackage{amsmath}
\usepackage[english]{babel}
\usepackage[utf8]{inputenc}
\usepackage{fancyhdr}
\textwidth 7.5in
\oddsidemargin -.5in
\topmargin -0.70in
\textheight 9.8in                      

\pagestyle{fancy}

%**************Fill in your ID and initials here*****************
\newcommand{\mc}[1]{\ensuremath{\mathcal{{#1}}}}
\newcommand{\mb}[1]{\ensuremath{\mathbb{{#1}}}}
\newcommand{\mf}[1]{\ensuremath{\mathfrak{{#1}}}}
\newcommand{\N}[1]{\ensuremath{\{1,\ldots,{#1}\}}}

\newcommand{\Worth}[1]{\{{#1} marks\}}
\newcommand{\Sln}{\smallskip \textbf{Solution.} }
\newcommand{\Extra}[1]{\{Extra credit: {#1} marks\}}


\setlength{\parskip}{0.15in}
\setlength{\parindent}{0in}


\newcommand{\NP}{\newpage \vspace*{-0.4in}}
\newcommand{\FP}{\vspace*{-0.6in}}
\newcommand{\tab}[1][1cm]{\hspace*{#1}}
\newcommand{\ES}{Erwin Schr\"odinger}

\lstset{ %
language=Java,
basicstyle=\ttfamily\scriptsize,commentstyle=\scriptsize\itshape,showstringspaces=false,breaklines=true}


\title{\huge PHYS234 Notes}
\author{Minyang Jiang}
\date{\today}

\begin{document}

\maketitle

\NP
\section{History}
\paragraph{A conservative revolutionary}
In about 1908, Planck convert to the view that the quantum of action representsan irreducible phenomenon beyond the understanding of classical physics
\paragraph{Einstein in 1905}
\begin{enumerate}
\item photoelectric effect
\item dissertation, proving the existence of atoms
\item Brownian motion
\item special relativity
\item $E=mc^2$
\end{enumerate}
\paragraph{Johann Jakob Balmer's formula}
$$v=R\left(\frac{1}{n_f^2}-\frac{1}{n_i^2}\right)$$
\paragraph{Rutherford}
atom model was unstable in classical physics
\paragraph{Niels Bohr}
- grandfather of quantum physics
\begin{enumerate}
\item solve the stability problem of Rutherford's model
\item classical physics could not apply inside the atom
\item orbits have something to do with the Planck - Einstein quantum relation of the light photon ($E=hv$).
\end{enumerate}
Bohr derived Balmer's formula
\paragraph{Bohr's model of atom}
\begin{enumerate}
\item Electrons in atoms orbit the nucleus
\item Electrons can only gain and lose energy by jumping from one allowed orbit to another, absorbing or emitting EM radiation with a frequency v given by the energy gap of the levels according to the Planck relation: $$\Delta E = E_2-E_1=hv$$
angular momentum L is restricted to be an integer multiple fo a fixed unit $$L=mvr=\frac{nh}{2\pi}=nh$$ where $n=1,2,3, \hdots$ is called the principal quantum number.\\
he mixed classical and quantum physics to get $$\frac{1}{\lambda}=R\left(\frac{1}{n_f^2}-\frac{1}{n_i^2}\right)$$
\end{enumerate}
\paragraph{Important applications of QT in 20\textsuperscript{th} century}
\begin{enumerate}
\item Invention of transistors
\item Invention of lasers
\item Invention of STM
\item ...
\end{enumerate}

\section{Chapter 1 The Wave Function and The Schr\"odinger Equation}
\subsection{de Broglie's matter wave}
\begin{equation}
p=\frac{h}{\lambda}
\end{equation}
This equation is valid for electrons, ions, photons and anyother $\Rightarrow$ every particle

Paul Langevin
de Broglie's thesis is the first feeble ray of light on  the worst of our plys enigmas
\begin{align*}
L&=n\hbar \tab (n=1,2,3,\hdots)\\
L_1 &= \hbar \neq 0 \Rightarrow r_0 = 0.527 \r{A}\\
2\pi r &= n \lambda \tab (n=1,2,3,\hdots)
\end{align*}
angular momentum:
$$L=rp=\frac{n\lambda}{2\pi}*\frac{h}{r}=n\frac{h}{2\pi}=n\hbar$$

\subsection{Schr\"odinger Equation}
\begin{equation}
F=-\frac{\partial V}{\partial x} \tab v-potential
\end{equation}
Classical phys: 
\begin{align*}
F&=ma \tab (Newtonian 2nd law)\\
F&=m\frac{d^2x}{dt^2}\\
\Downarrow\\
X&=X(t)\\
\Downarrow\\
v&=\frac{dx}{dt};p=mv=m\frac{dx}{dt}\\
T&=\frac{1}{2	}mv^2=\frac{1}{2}m(\frac{dx}{dt})^2\\
E&=T+V=\frac{p^2}{2m}+V
\end{align*}
Quantum mechanics:
$$E\rightarrow i\hbar \frac{\partial}{\partial t}; p\rightarrow \c{p}=i\hbar \triangledown$$
\begin{equation}
i\hbar \frac{\partial \Psi(x,t)}{\partial t} =\left( -\frac{\hbar^2}{2m}\triangledown^2+V \right)\Psi(x,t)
\end{equation}
$$\Psi(x,t) - \text{The Wave function} \Rightarrow \text{to describe physics properties}$$















\end{document}