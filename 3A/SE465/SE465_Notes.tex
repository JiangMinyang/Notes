\documentclass[10pt,usletter]{article}
\usepackage[letterpaper, hmargin=0.75in, vmargin=0.75in]{geometry}
\usepackage{graphicx}
\usepackage[hyphens]{url}
\usepackage{hyperref}
\usepackage{listings}
\usepackage{pgf}
\usepackage{courier}
\usepackage{amsfonts,amssymb,amsmath,amsthm,lastpage,fancyhdr,wrapfig,multirow}
\usepackage{palatino}
\usepackage{amsmath}
\usepackage{parallel,enumitem}
\usepackage{cancel}
\usepackage[english]{babel}
\usepackage{soul}
\usepackage[utf8]{inputenc}
\usepackage{fancyhdr}
\textwidth 7.5in
\oddsidemargin -.5in
\topmargin -0.70in
\textheight 9.8in                      

\pagestyle{fancy}

%**************Fill in your ID and initials here*****************
\newcommand{\mc}[1]{\ensuremath{\mathcal{{#1}}}}
\newcommand{\mb}[1]{\ensuremath{\mathbb{{#1}}}}
\newcommand{\mf}[1]{\ensuremath{\mathfrak{{#1}}}}
\newcommand{\N}[1]{\ensuremath{\{1,\ldots,{#1}\}}}

\newcommand{\Worth}[1]{\{{#1} marks\}}
\newcommand{\Sln}{\smallskip \textbf{Solution.} }
\newcommand{\Extra}[1]{\{Extra credit: {#1} marks\}}


\setlength{\parskip}{0.15in}
\setlength{\parindent}{0in}


\newcommand{\NP}{\newpage \vspace*{-0.4in}}
\newcommand{\FP}{\vspace*{-0.6in}}
\newcommand{\tab}[1][1cm]{\hspace*{#1}}

\lstset{ %
language=Java,
basicstyle=\ttfamily\scriptsize,commentstyle=\scriptsize\itshape,showstringspaces=false,breaklines=true}


\title{\huge SE465 Notes}
\author{Minyang Jiang}
\date{\today}

\begin{document}
\maketitle
\NP
\subsection{Example 1}
\begin{lstlisting}[language=Java]
static public int findLast(int[] x, int y) {
	for (int i = x.length - 1; i > 0; i--) {
		if (x[i] == y) {
			return i;
		}
	}
	return -1'
}
@Test
public void testFindlast() {
	int[] x = new int[] {2, 3, 5};
	assertEquals(0, FindLast.findLast(x, 2));
}
\end{lstlisting}
\begin{enumerate}
\item Identify and fix the fault \\
\tab for loop condition should be $i \geq 0$
\item If possible, identify a test case that does not exercise the fault \\
\tab x is null
\item if possible, identify a test case that exercise the fault, but no error state \\
\tab findLast([1, 2, 3], 2) will return -1
\item if possible, identify a test case that results in an error, but no failure \\
\tab trying to findsomething not there ([2], 5)
\item Identify the first error state
\end{enumerate}

\subsection{Example 2}
\begin{lstlisting}[language=Python]
class LineSegment:
	def __init__(self, x1, x2):
		self.x1 = x1; self.x2 = x2;
		
	def intersect(a, b):
		return (a.x1 < b.x2) & (a.x2 > b.x1);
\end{lstlisting}
Establishing correctness of intersect:
\begin{enumerate}
\item[•] case analysis of the inputs
\end{enumerate}
Other answers
\begin{enumerate}
\item[•] execute every statement of the unit under test
\item[•] feed random inputs
\item[•] check all outputs
\item[•] check values of each clause
\end{enumerate}
rename inputs: \\
$a = a.x_1 \tab b = b.x_1$\\
$A = a.x_2 \tab B = b.x_2$
\begin{enumerate}
\item[-]assume all points are distinct\\
\item[-]assume $a < b$ (we'll check both ways when constructing test cases) \\
\item[-]assume $a < A, b < B$
\end{enumerate}
aAbB\\
abAB\\
abBA
\begin{lstlisting}[language=Python]
# run this test as 'python line-intersection-test.py'

from line_intersection import *
import unittest

class TestIntersection(unittest.TestCase):
    def test_aAbB(self):
        a = LineSegment(0,2)
        b = LineSegment(3,7)
        self.assertFalse(intersect(a,b))
        self.assertFalse(intersect(b,a))

    def test_abAB(self):
        a = LineSegment(0,4)
        b = LineSegment(3,7)
        self.assertTrue(intersect(a,b))
        self.assertTrue(intersect(b,a))

    def test_abBA(self):
        a = LineSegment(0,4)
        b = LineSegment(1,2)
        self.assertTrue(intersect(a,b))
        self.assertTrue(intersect(b,a))

    def test_equality(self):
        a = LineSegment(0,2)
        b = LineSegment(2,4)
        self.assertTrue(intersect(a,b))        # A = b
        self.assertTrue(intersect(b,a))        # B = a
        a = LineSegment(2,2)
        b = LineSegment(0,4)
        self.assertTrue(intersect(a,b))        # a = A
        self.assertTrue(intersect(b,a))        # b = B
        a = LineSegment(0,2)
        b = LineSegment(0,4)
        self.assertTrue(intersect(a,b))        # a = b
        self.assertTrue(intersect(b,a))        # b = a

if __name__ == '__main__':
    unittest.main()

\end{lstlisting}

\subsection{•}
\noindent
\fbox{
\begin{minipage}[t]{0.48\linewidth}
Static:
\begin{enumerate}
\item[-] find faults\\
example:
\begin{enumerate}
\item type checking
\item dead code analysis
\end{enumerate}
\item[-] code inspection functionality and style
\item[-] program verification
\end{enumerate}
\end{minipage}}
\hfill%
\fbox{
\begin{minipage}[t]{0.48\linewidth}
Dynamic
\begin{enumerate}
\item[-] observe failures
\item[-] must generate inputs\\
\tab what are expected outputs?
\item[-] easy to run the program
\item[-] keywords\\
\tab white-box testing\\
\tab black-box testing
\end{enumerate}
\end{minipage}}

static techninques tradeoff:
\begin{enumerate}
\item[-] exhaustive 
\item[-] subject to false positives
\end{enumerate}

words I don't like \\
\tab \st{complete testing}\\
\tab \st{exhaustive testing}\\
\tab \st{full coverage}

First big question: When should I stop testing?
\begin{enumerate}
\item when I run out of time\\
\tab open-ended explorotroy testing\\
\tab for automatic input generation
\item when I'm close enough to being exhaustive\\
\tab explored enough (all) of \\
\tab[2cm] behaviours / use cases\\
\tab[2cm] program states\\
\tab[2cm] inputs\\
\tab[2cm] statements / branches
\end{enumerate}

\huge{observability, controlability}
\subsection{Coverage}
\normalsize
\begin{enumerate}
\item[-] idea: find reduced space + cover it with test cases
\end{enumerate}
Test Requirement (TR) - an element of an artifact that soe test case must satisfy

\section*{Infeasible Test Requirements}
unreachable code
definition: coverage level - Given a set of test requirements TR and a test set T, The \underline{coverage level} is the ratio of the number of TRs satisfied by T to the size of TR.

\subsection*{Exploratory Testing}
\begin{enumerate}
\item[-] usually carried out by testers
\item[-] unscripted in general\\
\tab "Exploratory teesting is simulatneous learning, test design, and test execution"
\end{enumerate}
\subsection*{Exploratory testing is good for}
\begin{enumerate}
\item[-] simulating actual use cases (realism)
\begin{enumerate}
\item[-] diversifying testing beyond scripts
\end{enumerate}
\item[-] finding single most important bug in sortest time
\item[-] being less siloed
\item[-] evaluating aparticular risk, see if scripted tests needed
\end{enumerate}
\subsection*{Exploratory Testing Process}
\begin{enumerate}
\item start with a goal /charter
\begin{enumerate}
\item[•]"Explore the product elements"
\end{enumerate}
\item decide which area of the software to test
\item design a test (informally)
\item execute test and log bugs
\item repeat as needed
\end{enumerate}
notes: don't produce exhaustive notes
output: \begin{enumerate}
\item set of bug reports
\item test notes (possibly a judgment)
\item artifacts (input, output)
\end{enumerate}
\subsection*{\underline{Results of WaterlooWorks Testing}}
\begin{enumerate}
\item sort order reversed on navigating to subsequent page
\item use of ophrase "current term" is ambiguous when choosing a term
\item on mobile, sort order not preserved when navigating to a job posting and going back
\end{enumerate}
\underline{Overall}
\begin{enumerate}
\item want some sort of judgment on overall usability of system\\
\tab do the primary functions work well enough
\end{enumerate}
\subsection*{Control flow graphs}
Coverage criteria for source code\\
stream of characters $\rightarrow$ stream of tokens $\rightarrow$ abstract syntax tree $\rightarrow$ control flow graph\\
\includegraphics[scale=0.1]{control_flow_graph_1}\\
\underline{Control Flow Graph}
\begin{enumerate}
\item representation of program which is easier to analyze\\
\tab Nodes: represent 0 or more statements\\
\tab Edge (directed) $(s_1, s_2)$ means $s_2$ may follow $s_1$ in an execution
\end{enumerate}


\includegraphics[scale=0.1]{control_flow_graph_2}\\

Basic Blocks: A \underline{basic block} has one entry point and one exit point and is maximal \\


\subsection{Statement and Branch coverage}
Definition: A \underline{test path} is a path p that starts at an initial node and ends at final node\\
Given a set of test requirements TR, for a graph criterian C, a test set T satisfies C on graph G iff for every test requirement tr in TR, at least one test path P in path (T) exists such that P satisfies tr\\
\underline{Test paths + test cases}
\begin{enumerate}
\item when testing , you provide test cases asa inputs to the program
\item each test case t gives at least one test path, path(t)
\item test set T = set of test cases path(t) = \{path(t) $\vert$ t $\in$ T\}
\end{enumerate}
Test Requirement:\\
\tab a condition that a test path may satisfy\\
Coverage criteria:\\
\tab generate sets of test requirements given graphs or other artifacts\\
Statement Coverage:\\
\tab have one test requirement for each node in the CFG\\
cause of nondetermminism:\\
\tab dependence on inputs, on thread scheduler, on memory addresses\\
Results: one input may give multiple outputs and multiple test cases

\underline{Review: Coverage Criteria}\\
criterian + graph \\
statement coverage $\Rightarrow$ set of test requirements\\

Test set = Bunch of inputs to SUT(system under test)\\
test cases t $\in$ T,\\
execution givs path (T), check how many test requirements path (T) satisfied\\

Criterion- Compute Path Coverage\\
TR contains all paths in G\\
Is infeasible when you have loops in G\\

\underline{Next: Finite- State Machines}\\
CFG: generate directly from source code\\
FSM: higher level, usually represent some design information\\
Example:\\
\tab navigation structure of webapp\\
Nodes: software states (variable values, or pages)\\
edges: transitions between software states (e.g. as triggered by the user with commands, or triggered by the passage of time)\\

\underline{Criterian Simple Round Trip Coverage (SRTC)}\\
TR contains at least one round-trip path for each reachable node in G that begins and ends a round-trip path\\
complete round-trip coverage: Like STRC, but "at least one" $\Rightarrow$ "all"\\
Defn A \underline{round-trip path} is a path of nonzero length with no internal cycles that starts and ends at the same node.\\

\underline{Deriving FSMs}\\
\begin{enumerate}
\item no good mechanical ways to get good FSMs
\item must understand the system\\
\tab will see some tools that help - comments 
\end{enumerate}
CFGs - poor way of getting FSMs\\
large + unwieldy\\

Sources of FSMs:
\begin{enumerate}
\item CFGs terrible
\item better: design docs / software structure	
\end{enumerate}
FSMs from software structure
\begin{enumerate}
\item requires lots of efforts
\item subjective
\item need to know the 
\end{enumerate}
FSMs from storeed state
\begin{enumerate}
\item identify key variables summarizing system state
\item values of variables determine state
\item changes in variables result in changes in state can be more mechanical than FSM from structure
\end{enumerate}

\underline{Generally:}\\
FSMs enable
\begin{enumerate}
\item test descriptions to be created before implementation FSMs easier to analyze than code
\end{enumerate}
But:
\begin{enumerate}
\item not exhaustive
\item doesn't necessarily match the implementation
\end{enumerate}

\underline{Syntax-based Testing}\\
\begin{enumerate}
\item input space grammars
\item mutation-based testing
\end{enumerate}

credit card numbers \\
Hover, there is a check sum, and many \#s conform to the regex but are semantically invalid\\
Idea: use the regexp to generate syntactically valid \#s then need to filter based on checksum to set valid \#s\\
Question: how to generate syntactically invalid id \#s?
\begin{enumerate}
\item modify the regexp to generate invalid \#s , e.g. super long \#s
\item change somthing while generating the string
\end{enumerate}

Instead of regex, can use context free grammars using grammears:
\begin{enumerate}
\item generate vallid and invalid program inputs
\item inside program validate input
\end{enumerate}

Generators:\\
Start with start production\\
replace non terminals with what's on RHS eventually set strings belonging\\
repeat until you have enough input language

\end{document}