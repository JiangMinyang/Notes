\documentclass[10pt,usletter]{article}
\usepackage[letterpaper, hmargin=0.75in, vmargin=0.75in]{geometry}
\usepackage{graphicx}
\usepackage[hyphens]{url}
\usepackage{hyperref}
\usepackage{listings}
\usepackage{pgf}
\usepackage{courier}
\usepackage{amsfonts,amssymb,amsmath,amsthm,lastpage,fancyhdr,wrapfig,multirow}
\usepackage{palatino}
\usepackage{amsmath}
\usepackage[english]{babel}
\usepackage[utf8]{inputenc}
\usepackage{fancyhdr}
\textwidth 7.5in
\oddsidemargin -.5in
\topmargin -0.70in
\textheight 9.8in                      

\pagestyle{fancy}

%**************Fill in your ID and initials here*****************
\newcommand{\mc}[1]{\ensuremath{\mathcal{{#1}}}}
\newcommand{\mb}[1]{\ensuremath{\mathbb{{#1}}}}
\newcommand{\mf}[1]{\ensuremath{\mathfrak{{#1}}}}
\newcommand{\N}[1]{\ensuremath{\{1,\ldots,{#1}\}}}

\newcommand{\Worth}[1]{\{{#1} marks\}}
\newcommand{\Sln}{\smallskip \textbf{Solution.} }
\newcommand{\Extra}[1]{\{Extra credit: {#1} marks\}}


\setlength{\parskip}{0.15in}
\setlength{\parindent}{0in}


\newcommand{\NP}{\newpage \vspace*{-0.4in}}
\newcommand{\FP}{\vspace*{-0.6in}}
\newcommand{\tab}[1][1cm]{\hspace*{#1}}

\lstset{ %
language=Java,
basicstyle=\ttfamily\scriptsize,commentstyle=\scriptsize\itshape,showstringspaces=false,breaklines=true}


\title{\huge CS370 Notes}
\author{Minyang Jiang}
\date{\today}

\begin{document}

\maketitle

\NP
\section{Floating Point Number Systems}
\subsection{Introduction}
$$F(\beta, t, L, u)$$
continas:
\begin{center}
0 or $\pm 0.\beta_1\beta_2...\beta_t * \beta^d$ \\
where $\beta_1 \neq 0$ \\ 
$0 \leq \beta_i \le \beta$ \\
$L \leq d \leq u$
\end{center}
Two common systems used today - Base 2
\begin{enumerate}
\item Single precision: $F(2, 24, -126, 127)$ 
\item Double precision: $F(2, 53, -1022, 1023)$ 
\end{enumerate}
Important concepts\\
If $x$ is any real number then set $fl(x)=$ floating point representation of x \\
if we write: $$x=\pm 0.x_1x_2x_3...x_tx_{t+1}...*\beta^d$$
$$fl(x)=\pm 0.x_1x_2x_3...x_t*\beta^d$$
relative error $$\delta_x = \frac{fl(x)-x}{x}$$
$$\|\delta_x\| \leq ?$$
\begin{align*}
\frac{\|fl(x)-x\|}{\|x\|}&= \frac{0.00\hdots0x_{t+1}\hdots*\beta^d}{0.x_1x_2\hdots x_{t+1}\hdots*\beta^d} \\
&=\frac{0.x_{t+1}x_{t+2}\hdots*\beta^{-t}}{x_1.x_2\hdots * \beta_{-1}}\\
\delta &= \frac{fl(x)-x}{x} \tab \text{then } \|\delta\| \leq \epsilon =
\begin{cases}
\beta^{1-t} \\
\frac{\beta^{1-t}}{2} \
\end{cases}\\
fl(x)&=x(1+\delta) \tab |\delta| \le \epsilon \\
\end{align*}
What about floating point arithmetic? \\
\tab x, y  real numbers, $x+y$ real\\
\tab $x\oplus y = $addition inside floating pt system $=fl(fl(x) + fl(y))$

\subsection{Analysis some errors in computation}
Example. Addition
\begin{align*}
\left\|\frac{(x+y)-(x\oplus y)}{x + y}\right\|&=\frac{\|(x+y)-fl(fl(x)+fl(y))\|}{x+y}=\frac{\|x+y-x(1+\delta)+y(1+\delta)\|}{x+y}\\
&=\frac{\|x+y-(x+y+\delta_1 x+\delta_2 y + x\delta_3 + y\delta_3 + \delta_1\delta_3 x+ \delta_2 \delta_3 y)\|}{x+y}\\
&\leq \frac{\|\delta_1 x\|+\|\delta_2 y\| + \|\delta_3 x\| + \|\delta_3 y\| + \delta_1\delta_3 x + \|\delta2\delta_3 y\|}{\|x+y\|}\\
&\le \frac{(\|x\|+\|y\|)(2\epsilon + \epsilon^2)}{\|x+y\|} \\
\|\delta_1\| \le \epsilon,\tab &\|\delta_2\| \le \epsilon,\tab \|\delta_3\| \le \epsilon
\end{align*}
$\Rightarrow$ if x and y have same sign then relative error of addition $$\left\|\frac{x \oplus y - (x+y)}{x+y}\right\| \le 2\epsilon + \epsilon^2$$
However if x and y have opposite sign, then you potentially have a problem particularly when $x+y \approx 0$, Situation is called \textbf{Catastrophic cancellation}
\begin{align*}
x&= 0.x_1x_2\hdots x_{t-1} x_t x_{t+1}\hdots * \beta^d\\
y&=-0.x_1x_2\hdots x_{t-1} x_t x_{t+1}\hdots * \beta^d\\
x+y&= 0.00\hdots0??\hdots * \beta^d\\
\end{align*}

\subsection{How about some algorithms?}
\subsubsection{Example}
Given $\alpha$, compute: $$I_n=\int_0^1\frac{x^n}{x+\alpha}dx \tab n=0,1,\hdots,100\hdots$$
\underline{Step 1} $$I_0=\int_0^1 \frac{1}{x+\alpha}dx=Ln(x+\alpha)=Ln(1+\alpha)-Ln(\alpha)=Ln\left(\frac{1+\alpha}{\alpha}\right)$$
e.g.
\begin{align*}
\alpha = 0.5 \text{ then } I_0=1.098612288668\hdots\\
\alpha = 2.0 \text{ then } I_0=0.405465108108\hdots\\
\end{align*}
\underline{Step 2} \tab Notice: $$I_{n+1}=\int_0^1\frac{x^{n+1}}{x+\alpha}dx=\int_0^1\frac{x^n(x+\alpha-\alpha}{x+\alpha}dx=\int_0^1x^ndx-\alpha\int_0^1\frac{x^n}{x+\alpha}dx$$
$$I_{n+1}=\frac{1}{n+1}-\alpha I_n$$
\begin{align*}
I_0&\\
I_1 &= 1-\alpha I_0\\
I_2 &= \frac{1}{2} - \alpha I_1\\
\vdots\\
I_{100} &= \frac{1}{100} - \alpha I_00
\end{align*}
If $\alpha = 0.5$ then $I_{100} = 0.00664$\\
if $\alpha = 2.0$ then $I_{100} = 2.1 * 10^{22}$\\
$$\|I_n\| \leq \frac{1}{1+\alpha}$$
Let's analyze what is happening \\
\underline{Math}: $$I_0^{ex}, I_{n+1}^{ex}=\frac{1}{n+1}-\alpha I_n^{ex}$$ 
\underline{CS}:	$$I_0^{app}, I_{n+1}^{app}=\frac{1}{n+1}-\alpha I_n^{aop}$$
At every step, therer is some error:
\begin{align*}
e_0&=I_0^{ex}-I_0^{app} \\
e_n&=I_0^{ex}-I_n^{app}
\end{align*}
\underline{Notice:} 
\begin{align*}
e_{n+1}&=I_{n+1}^{ex}-I_{n+1}^{app} \\
&=\left(\frac{1}{n+1}-\alpha I_n^{ex}\right)-\left(\frac{1}{n+1}-\alpha I_n^{app}\right) \\
&=-\alpha I_n^{ex}+\alpha I_n^{app} \\
&=-\alpha \left(I_n^{ex}-I_n^{app}\right) \\
&=-\alpha e_n \\
&=(-\alpha)^{n+1}e_0
\end{align*}
If $\|\alpha\|\le 1$ then $\|e_n\|\rightarrow 0$ as $n\rightarrow \inf$\\
If $\|\alpha\|\ge 1$ then $\|e_n\|\rightarrow \inf$ as $n\rightarrow \inf$\\
What to do when $\|\alpha\| \ge 1$:
\begin{align*}
I_n&=\frac{1}{\alpha (n+1)} - \frac{1}{\alpha} I_{n+1}\\
e_n&=-\frac{1}{\alpha}e_{n+1}
\end{align*}
If $\|\alpha\| \ge 1$ then work backwards, e.g. $I_{100}$, Do $I_{200}, I_{199}\hdots$


















\end{document}
